\documentclass[a4paper,11pt]{article}

\usepackage{amsmath}
\usepackage{tikz}
\usetikzlibrary{arrows,automata,shapes}

\begin{document}

\title{On Bulks and Sculptures}

\author{Uli Fahrenberg}

\maketitle

When talking about sculptures and bulks, Christopher and I found out
that the definitions seem to be wrong.

The intuition of sculpting is ``take a cube of some dimension, and
sculpt at it, taking off bits and pieces here and there''.  Hence
bulks are ``geometric'' cubes; but they can't be combinatorial cubes,
because you can't take bits and pieces away from combinatorial cubes.
Rather, bulks should be \emph{subdivisions} of combinatorial cubes.

Figure~\ref{fi:bulk} shows a two-dimensional example.  On the left,
the bulk plus the pieces we want to sculpt away (in red); on the
right, the subdivision we need for the combinatorics.  The resulting
scuplture consists of 17 squares (plus their 1- and 0-dimensional
faces).

\begin{figure}[bp]
  \centering
  \begin{tikzpicture}[scale=.9]
    \begin{scope}
      \draw (0,0) to (5,0) to (5,5) to (0,5) to (0,0);
      \draw[red] (-.5,3) to (1,3) to (1,4) to (-.5,4);
      \draw[red] (3,4) to (4,4) to (4,4.5) to (3,4.5) to (3,4);
      \draw[red] (4,-.5) to (4,1) to (5.5,1);
    \end{scope}
    \begin{scope}[xshift=18em]
      \draw (0,0) to (5,0) to (5,5) to (0,5) to (0,0);
      \draw[dashed] (0,1) to (5,1);
      \draw[dashed] (0,3) to (5,3);
      \draw[dashed] (0,4) to (5,4);
      \draw[dashed] (0,4.5) to (5,4.5);
      \draw[dashed] (1,0) to (1,5);
      \draw[dashed] (3,0) to (3,5);
      \draw[dashed] (4,0) to (4,5);
      \draw[red] (-.5,3) to (1,3) to (1,4) to (-.5,4);
      \draw[red] (3,4) to (4,4) to (4,4.5) to (3,4.5) to (3,4);
      \draw[red] (4,-.5) to (4,1) to (5.5,1);
    \end{scope}
  \end{tikzpicture}
  \caption{
    \label{fi:bulk}
    Sculpting}
\end{figure}

Hence the following definitions (where an \emph{$n$-cube} is an image
of the Yoneda embedding):
\begin{itemize}
\item A \emph{bulk} is a subdivided $n$-cube.
\item A \emph{sculpture} is any subset of a bulk.
\end{itemize}

Now these definitions are problematic, because they are geometric in
nature.  We want combinatorial definitions, and subdivisions are not
combinatorial (they're not precubical morphisms).

So we'd like definitions of the form ``A precubical set $X$ is a bulk
if \emph{blabla}''; ``A precubical set $X$ is a sculpture if
\emph{blibli}''.

We've talked with the local expert, Emmanuel Haucourt, about this: It
should be easy enough to write up what ``\emph{blabla}'' is; but there
is nobody who seems to know the precise ``\emph{blibli}''.

Note the famous example of the ``broken box'' (Fig.~4 in
\texttt{main3}): we proved in Oslo that it's not a sculpture in the
old sense, but this proof doesn't work anymore.  It might well be a
sculpture in the new (and better!) sense.

There is a notion of \emph{loop-free} for precubical sets; essentially
it means that for any $x\in X$, once you leave $x$, you never see $x$
again.  Sculptures are obviously loop-free, but is this enough?

We'll try to work this out; this is fun!


\end{document}
