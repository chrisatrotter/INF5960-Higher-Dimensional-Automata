\section{Summary}
    \label{sec:traditional-concurrency-summary}

    When considering the classification of concurrency models of Winskel et al. \cite{winskel94RelationshipsConcurrency}, we have that models are either simulating concurrency, known as \emph{interleaving}, or dealing directly with concurrency, namely \emph{non-interleaving}. The difference being that mutually exclusive occurrences are not distinguishable from truly independent ones in the former, but distinguishable in the latter.
    
    Interleaving models are not able to distinguish mutually exclusive events from truly independent events because the models depend on which events a process takes to be \emph{atomic} \cite[page 4]{Pratt86pomsets}. When we say atomic, we mean an indivisible action that must complete without interruption, in other words be considered instantaneous. If we suppose that an atomic event can be \emph{refined} in the same manner as how Pratt explains refinement of atomic actions in \cite{pratt91hda}, then the interleaving model can be seen to not have interleaved subactions. We will consider such a refinement as \emph{action refinement}. Action refinement is the method of developing a system by starting from a abstract specification, and gradually refining its action by providing more details \cite{Johansen16DecEventBasedConcurrencyRefinement, GlabbeekG89refinement}.
    
    % In the literature, it has shown that interleaving models are not well behaved under action refinement. \cite{x,y,z}.
    In the literature, it has been shown that interleaving models are not well behaved under action refinement \cite{GlabbeekG89refinement, GlabbeekG01refinement}. The model is forced to commit itself in advance to a particular level of granularity. For example, a programmer may give details on actions that, at the time, are regarded as atomic. However, at a future release the programmer may discover that the actions have substructures, see \cite[Example 1.1]{GlabbeekG89refinement}. The programmer cannot refine the action since it is atomic. There might be occurrences where the interleaving model can be seen not to have interleaved subactions. The failure  to interleave subactions is not characteristic of non-interleaving concurrency, but rather of a hidden assumption of excluded middle, or \emph{mutual exclusion} as it is more commonly known in concurrent processes \cite{Pratt00Sculptures}. We consider mutual exclusion to be where no actions can happen simultaneously, and must occur after each other.
    
    From our definition of transition systems, we are only able to simulate the parallel execution of two actions as the "\emph{interleaving}" of $ab$ or $ba$, see Figure \ref{fig:transition-system}. In the next chapter we will introduce asynchronous transition systems which extends the transition system to include an independence relation that is equipped to distinguish the mutual exclusion and independent occurrence of two events.