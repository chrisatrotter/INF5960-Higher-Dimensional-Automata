    For the most part, early models of computation, such as Turing Machines \cite{turing_1937}, are sequential, meaning that there is a strict order in which computational steps happen. The strict ordering ensures that the sequence of computational steps are unambiguous, such that one could provide a single history of the executed computation. For sequential computation, the $\lambda$-calculus has become a fundamental unifying formalism. This framework is also accommodated by a rich variety of developments such as types and higher-type computation, program logic, operational semantics, and denotational semantics. However, concurrent computation has yet to see a unifying framework in the way that $\lambda$-calculus is for sequential computation.
    
    There has been extensive research within the field of concurrency theory to approach a unifying concept, and an outline of many deep, fundamental results was done by Garavel in \cite{gravel08summary}. In this thesis, we will investigate one of the opportunities mentioned by Garavel and present the notion of sculpting for higher-dimensional automata, as a result. The opportunity described was the idea of having models everywhere and a way to transform one model into another. Specifically, models for different purposes and a way to translate between them. Considering this opportunity would unify several pre-existing concepts such as translation and refinement.

    Since the 1960s, a wide variety of models of concurrency have been proposed. One of the first proposals was Petri nets. During the past decades, there have been other proposals for models of concurrency, such as transition systems, higher dimensional automata and event structures. Even though the nature of these models are very different, they have shown to have a tight relation to each other, as presented by Winskel \cite{winskel95modelsCategory} and Goubault \cite{Goubault18RelationshipsModelsForConcurrency}. Because of this relation, we classify models of concurrency as state-based, \emph{automata}, or event-based models, \emph{schedules}. Also, we further classify these as interleaving or non-interleaving models. For sequential behaviour these classes are considered equivalent. In this thesis, we will  focus on the classification of interleaving versus non-interleaving models. We will first review \emph{transition systems} as an interleaving model and show how they fall short of our intuition of concurrent behaviour.