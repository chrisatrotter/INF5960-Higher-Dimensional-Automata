\documentclass[tikz,border=10pt]{standalone}
\usepackage{tikz}
\usepackage{tikz-cd}
\usetikzlibrary{arrows,automata,shapes,positioning,decorations.pathmorphing}
% \tikzset{->,>=stealth',auto}
\tikzset{->,auto}
\tikzset{>={Latex[width=2mm,length=2mm]}}
\tikzset{state/.style={shape=circle, draw, fill=white, initial text=,
    inner sep=.5mm, minimum size=2mm}}
\tikzset{state with output/.style={shape=rectangle split, rectangle
    split parts=2, draw, fill=white,
    initial text=, inner sep=1mm}}
\tikzset{every node={font=\footnotesize}}
\begin{document}
  \begin{tikzpicture}[node distance=3cm, align=center]
  \tikzstyle{every node}=[font=\footnotesize]
    \tikzstyle{every state}=[fill=white,shape=circle,inner sep=.5mm,minimum size=3mm]
    \title{A transition system}

    \node[state](q1) [label=left:{$s_{1}$}]                        {};
    \node[state](q2) [above right of=q1, label=above:{$s_{2}$}]    {};
    \node[state](q3) [below right of=q1, label=below:{$s_{3}$}]    {};
    \node[state](q4) [below right of=q2, label=right:{$s_{4}$}]    {};
    
    \draw [->][draw=black] (q1) to node [above left] {$b$} (q2);
    \draw [->][draw=black] (q1) to node [below left] {$a$} (q3);
    \draw [->][draw=black] (q2) to node [above right] {$a$} (q4);
    \draw [->][draw=black] (q3) to node [below right] {$b$} (q4);
    
  \end{tikzpicture}
\end{document}

