    
    In this chapter we define a method of modelling concurrent behaviour of higher-dimensional automata. We call this method \emph{sculpting}. In other words, the process of modelling a concurrent system using higher-dimensional automata may be considered as a sculpting process $-$ take one single higher-dimensional cube, having enough concurrency, meaning enough events, and remove cells until the desired concurrent behaviour is obtained. The model obtained is called a \emph{sculpture}.
    
    We investigate the method of sculpting which has not been studied for higher-dimensional automata before. One goal is to tighten the correlation between ST-structures and higher-dimensional automata, which was left open in \cite{Johansen16STstruct} where neither model could be embedded into the other. The main result of our study is the fact that the sculpting method cannot build all higher-dimensional automata, but only a strict subset of these. We identify the category of sculptures and its relation to the category of higher-dimensional automata and the category of ST-structures.