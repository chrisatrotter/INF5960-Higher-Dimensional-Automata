\section{Sculptures}
\label{sec:sculptures}
    The intuition for the method of programming as sculptures was first introduced by Pratt in \cite{Pratt00Sculptures} as a method of identifying events in higher dimensional automata. It was considered in regards to the Chu space approach to higher dimensional automata, where one instead starts with a single cube of very large, possibly infinite, dimension and "\emph{sculpts}" the desired process by removing unwanted faces. Initially, a cube constitutes the events to be a discrete and unstructured set, forming a complete cube. By removing states, the event set becomes structured. For example, sculpting two events by removing all the states distinguishing them renders them equivalent. The removal of states from higher-dimensional automata may be understood as furnishing higher-dimensional automata with a certain structure.
    
    A sculpture consists of a higher-dimensional automaton, a bulk and an embedding. A \emph{bulk} is a \emph{non-selflinked} precubical set which is generated by a single cube of very large, possibly infinite, dimension. A precubical set $\mathcal{X}$ is \emph{non-selflinked} if it holds for all $x, y\in \mathcal{X}$ that there exists at most one index sequence $i_1\le\dotsc\le i_n$ such that $x= \alpha^1_{ i_1}\dotsm \alpha^n_{ i_n} y$ for $\alpha^1,\dotsc, \alpha^n\in\{ s, t\}$. Hence, there is unique representation for each cell in the precubical set. An embedding is an injective morphism such that there is a mapping from the precubical set to the bulk. Intuitively, an embedding is considered to be the same as sculpting the desired process by removing unwanted faces.

    \begin{definition}[Bulk]\label{def:bulk}
        A precubical set $\mathcal{X}$ is a \emph{bulk} if it is non-selflinked and generated by precisely one $n$-cube, that is, if $\num{ \mathcal{X}_{ \dim X}}= 1$.
    \end{definition}

    Any two $d$-dimensional bulks are isomorphic, where the isomorphism is generated by a permutation of the $d$ directions of the generating cubes.  Hence we may talk of \emph{the} $d$-dimensional bulk, and denote it by $\bulk{d}$.

    We develop a naming scheme for bulks inspired by Chu spaces. 

    \begin{definition}[Canonical naming]
    \label{def:canonical-naming}
        Fix $d\in \Nat$ and let $\bulk{d}= \{ 0, \executing, 1\}^d$. For $n= 0,\dotsc, d$, let $B_n=\{( x_1,\dotsc, x_d)\in B\mid \num{\{ i\mid x_i= \executing\}}= n\}$ be the set of tuples with precisely $n$ occurrences of $\executing$.  For $n= 1,\dotsc, d$, define face maps $s_k, t_k: B_n\to B_{ n- 1}$ ($k= 1,\dotsc, n$) as follows: for $x=( x_1,\dotsc, x_d)\in B_n$ with $x_{ i_1}=\dotsm= x_{ i_n}= \executing$, let $s_k x=( x_1,\dotsc, 0_{ i_k},\dotsc, x_d)$ and $t_k x=( x_1,\dotsc, 1_{ i_k},\dotsc, x_d)$ be the tuples with the $k$-th occurrence of $\executing$ set to $0$ or $1$, respectively. We call this the \emph{canonical naming} for the bulk $\bulk{d}$.
    \end{definition}

    The above construction essentially labels the $d$-bulk with lists of Chu-labels, namely, with states from the Chu space on $d$ events. Any cell $q^{b}$ in a bulk $\bulk{d}$ is reached from the highest cell $q_{d}$ through a sequence of face maps applications $\alpha_{i_{1}}\dotsm \alpha_{i_{k}}q_{d}=q^{b}$ of correct indices. 

    \begin{lemma}
        \label{lemma:chu-labelling-bulk}
        The structure $\bulk{d}$ as defined above is the $d$-dimensional bulk.
    \end{lemma}

    \begin{proof}
        It is trivial to check that $B$ is a precubical set.  $\bulk{d}$ is also non-selflinked, and $\num{ B_d}= \num{\{( \executing,\dotsc, \executing)\}}= 1$, thus by uniqueness up to isomorphism, $\bulk{d}$ is the $d$-dimensional bulk.
    \end{proof}

    The \emph{initial state} $\initbulk{d}$ of the bulk $\bulk{d}$ is the cell named $( 0,\dotsc, 0)$ in the canonical naming.  (any automorphism of $\bulk{d}$ fixes this cell.)  This turns bulks into $\HDA$.

    The $d$-dimensional bulk can be embedded into any bulk of dimension $d'\ge d$ by using the embedding $\embedMorphism^{d'}_{d}:\bulk{d}\hookrightarrow \bulk{d'}$ which maps any cell $( t_1,\dotsc, t_d)$ to $( t_1,\dotsc, t_d, 0,\dotsc, 0)$ in the canonical naming.  It can easily be shown that up to isomorphism, $\embedMorphism^{ d'}_d$ is the \emph{only} $\HDA$ morphism from $\bulk{ d}$ to $\bulk{ d'}$, hence also that there are no $\HDA$ morphisms $\bulk{d}\to \bulk{d'}$ for $d'< d$.

    \begin{definition}[Sculpture]
        \label{def:sculptures}
        A \emph{sculpture} is a $\HDA$ $\mathcal{X}$ together with a bulk $\bulk{d}$ and a $\HDA$ embedding morphism, in other words, an injective morphism, $\embedMorphism: X\hookrightarrow \bulk{d}$. We denote a sculpture by ($\mathcal{X}, \bulk{d}, \embedMorphism$).

        \emph{morphism} of sculptures $\embedMorphism: \mathcal{X} \hookrightarrow \bulk{ d}$, $\embedMorphism': \mathcal{X}' \hookrightarrow \bulk{ d'}$ is a pair of $\HDA$ morphisms $f: \mathcal{X} \to \mathcal{X}'$, $b: \bulk{ d}\to \bulk{ d'}$ such that the square

        \begin{equation*}
            \xymatrix{%
            \mathcal{X} \ar[d]_{\embedMorphism} \ar[r]_f & \mathcal{X}' \ar[d]^{\embedMorphism'}
            \\ \bulk{d} \ar[r]^b & \bulk{d'}
            }
        \end{equation*}

        commutes, namely, $b\circ \embedMorphism= \embedMorphism'\circ f$.  By the above considerations, this entails that $d'\ge d$ and $b$ is injective, hence also $f$ must be injective.  To sum up, morphisms of sculptures are commuting squares of embeddings. Two sculptures are isomorphic, denoted $\isomorphic$, when the morphism between them has both $f,b$ bijections; thus making the bulks of the same dimension.
    \end{definition}

    \begin{remark}
        One precubical set can be seen as two different sculptures, for example, from two different dimensional bulks, in both cases being a simplistic sculpture, meaning it all depends on the embedding morphism. Then this precubical set enters as source object of several sculpture morphisms, as seen in Figure~\ref{fig:asymmetric-conflict-st-hda}.
    \end{remark}
    
    The major problem with both translations is related to the fact that it is not clear how to identify the events in a higher-dimensional automaton. The best example for this is the fact that the mapping ST destroys the interleaving square, which is due to the fact that the standard method of identifying events in a higher-dimensional automaton by equivalent transitions opposite in a filled square fails for this unfilled square; see Figure \ref{fig:HDA-broken-box} (left). This same issue about events is also the one that causes the problem for the other mapping H where we could not say in the higher-dimensional automaton that was generated whether this was representing two or three events.
    
    However, we show how sculpting allows to identify the events in a higher-dimensional automaton in the same way as ST-structures work with events (we still use the standard intuitive method of seeing events as equivalence classes of transitions opposite in a filled square).
    
    We show that the ST-structures captures precisely the higher-dimensional automaton that can be seen as sculptures. 

