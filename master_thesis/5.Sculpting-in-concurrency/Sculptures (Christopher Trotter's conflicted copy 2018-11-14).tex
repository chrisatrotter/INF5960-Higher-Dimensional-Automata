\section{Sculptures}
\label{sec:sculptures}
    The intuition for the method of programming as sculptures was first introduced by Pratt in \cite{Pratt00Sculptures} as a method of identifying events in higher dimensional automata. It was considered in regards to the Chu space approach to higher dimensional automata, where one instead starts with a single cube of very large, possibly infinite, dimension and "\emph{sculpts}" the desired process by removing unwanted faces. Initially, a cube constitutes the events to be a discrete and unstructured set, forming a complete cube. By removing states, the event set becomes structured. For example, sculpting two events by removing all the states distinguishing them renders them equivalent. The removal of states from higher-dimensional automata may be understood as furnishing higher-dimensional automata with a certain structure.
    
    %This is the view of a state as a morphism from A to 3 respecting whatever structure is imputed to A by the omission of certain states. While in many cases such structure may be independently characterized in more conventional ways, for example, with $\leq$ and $\#$ as with prime coherent event structures. 
    
    %"\emph{Sculpting}" approach to specifying structure has three important benefits: simplistic, generality and categorical algebra. This last arises by formulating acyclic higher-dimensional automata as triadic Chu spaces, inheriting their notion of morphism $f:(E,X) \rightarrow (A,Y)$ as a function $f: E \rightarrow A$ that is continuous in the sense that for all $y \in Y, y \circ f \in X$, $- \circ f$ being the inverse-image function $f^{-1}$.
    
<<<<<<< HEAD
    A sculpture consists of a higher-dimensional automaton, a bulk and an embedding. A \emph{bulk} is a \emph{non-selflinked} precubical set which is generated by a single cube of very large, possibly infinite, dimension. A precubical set $\mathcal{X}$ is \emph{non-selflinked} if it holds for all $x, y\in \mathcal{X}$ that there exists at most one index sequence $i_1\le\dotsc\le i_n$ such that $x= \alpha^1_{ i_1}\dotsm \alpha^n_{ i_n} y$ for $\alpha^1,\dotsc, \alpha^n\in\{ s, t\}$. Hence, there is unique representation for each cell in the precubical set. An embedding is an injective morphism such that there is a mapping from the precubical set to the bulk. Intuitively, an embedding is considered to be the same as sculpting the desired process by removing unwanted faces. As a notation, we write $\num A$ for the number of elements of a set $A$. 
=======
    A sculpture consists of a higher-dimensional automaton, a bulk and an embedding. A \emph{bulk} is a \emph{non-selflinked} precubical set which is generated by a single cube of very large, possibly infinite, dimension. A precubical set $\mathcal{X}$ is \emph{non-selflinked} if it holds for all $x, y\in \mathcal{X}$ that there exists at most one index sequence $i_1\le\dotsc\le i_n$ such that $x= \alpha^1_{ i_1}\dotsm \alpha^n_{ i_n} y$ for $\alpha^1,\dotsc, \alpha^n\in\{ s, t\}$. Hence, there is unique representation for each cell in the precubical set. An embedding is an injective morphism such that there is a mapping from the precubical set to the bulk. Intuitively, an embedding is considered to be the same as sculpting the desired process by removing unwanted faces.
>>>>>>> 6a85ddbdff57f2fe030ed38810e4ef49c2b1c3cd
    
    
    
    
    % The method of sculpting was intuitively provided by Pratt

    % Transition and Cancellation in Concurrency and Branching time - Pratt 2003
    %The removal of states from $3^{A}$ may be understood as furnishing $3^{A}$ with a certain structure, namely that structure which is disrespected by precisely the removed states.
    
    %This is the view of a state as a morphism from A to 3 respecting whatever structure is imputed to A by the omission of certain states. While in many cases such structure may be independently characterized in more conventional ways, for example, with $\leq$ and $\#$ as with prime coherent event structures,
    
    %this subsetting, or "\emph{sculpture}" approach to specifying structure has three important benefits: simplistic, generality and categorical algebra. This last arises by formulating acyclic HDAs as triadic Chu spaces, inheriting their notion of morphism $f:(A,X) \rightarrow (B,Y)$ as a function $f: A \rightarrow B$ that is continuous in the sense that for all $y \in Y, y \circ f \in X$ ($- \circ f$ being the inverse-image function $f^{-1}$).

    %%%% Uli and Christian paper
    
    %\ulilong{Introduce sculptures.  Define a proper category of sculptures as an arrow category in HDA.  Isomorphism of sculptures.  Show that sculptures are Euclidean complexes. Lemma: for any two points in a sculpture, any two paths between them have the same length. Use this to show that the $ab+c$ triangle is not a sculpture; but say that, of course, its unfolding is a sculpture.}

    %A precubical set $X$ is \emph{non-selflinked} if it holds for all $x, y\in X$ that there exists at most one pair of index sequences $( i_1,\dotsc, i_n, \nu_1,\dotsc, \nu_n)$ with $i_k\le i_\ell$ for $k\le \ell$ for which $x= \delta_{ i_1}^{ \nu_1}\dotsm \delta_{ i_n}^{ \nu_n} y$.

    % That is to say, $X$ is non-selflinked iff any $x\in X$ is \emph{embedded} via the Yoneda morphism, hence iff all $x$'s iterated faces are genuinely different, as opposed to identified with each other.  This implies that $X$ have no ``short loops'': elements $x\in X$ for which $\delta_i^0 x= \delta_i^1 x$ for some $i$.  It does not exclude ``long'' loops.


    \begin{definition}[Bulk]\label{def:bulk}
        A precubical set $\mathcal{X}$ is a \emph{bulk} if it is non-selflinked and generated by precisely one $n$-cube, that is, if $\num{ \mathcal{X}_{ \dim X}}= 1$.
    \end{definition}

    Any two $d$-dimensional bulks are isomorphic, where the isomorphism is generated by a permutation of the $d$ directions of the generating cubes.  Hence we may talk of \emph{the} $d$-dimensional bulk, and denote it by $\bulk{d}$.

    We develop a naming scheme for bulks inspired by Chu spaces. 

    \begin{definition}[Canonical naming]
    \label{def:canonical-naming}
        %Fix $d\in \Nat$ and let $\bulk{d}= \{ 0, \executing, 1\}^d$.  For $n= 0,\dotsc, d$, let $B_n=\{( x_1,\dotsc, x_d)\in B\mid \num{\{ i\mid x_i= \executing\}}= n\}$ be the set of tuples with precisely $n$ occurrences of $\executing$. For $n= 1,\dotsc, d$, define face maps $\delta_k^\nu: B_n\to B_{ n- 1}$ ($k= 1,\dotsc, n$, $\nu= 0, 1$) as follows: for $x=( x_1,\dotsc, x_d)\in B_n$ with $x_{ i_1}=\dotsm= x_{ i_n}= \executing$, let $\delta_k^\nu x=( x_1,\dotsc, \nu_{ i_k},\dotsc, x_d)$ be the tuple with the $k$-th occurrence of $\executing$ set to $\nu$. We call is the \emph{canonical naming} for the bulk $\bulk{d}$.
        
        Fix $d\in \Nat$ and let $\bulk{d}= \{ 0, \executing, 1\}^d$. For $n= 0,\dotsc, d$, let $B_n=\{( x_1,\dotsc, x_d)\in B\mid \num{\{ i\mid x_i= \executing\}}= n\}$ be the set of tuples with precisely $n$ occurrences of $\executing$.  For $n= 1,\dotsc, d$, define face maps $s_k, t_k: B_n\to B_{ n- 1}$ ($k= 1,\dotsc, n$) as follows: for $x=( x_1,\dotsc, x_d)\in B_n$ with $x_{ i_1}=\dotsm= x_{ i_n}= \executing$, let $s_k x=( x_1,\dotsc, 0_{ i_k},\dotsc, x_d)$ and $t_k x=( x_1,\dotsc, 1_{ i_k},\dotsc, x_d)$ be the tuples with the $k$-th occurrence of $\executing$ set to $0$ or $1$, respectively. We call this the \emph{canonical naming} for the bulk $\bulk{d}$.
    \end{definition}



    The above construction essentially labels the $d$-bulk with lists of Chu-labels, namely, with states from the Chu space on $d$ events. Any cell $q^{b}$ in a bulk $\bulk{d}$ is reached from the highest cell $q_{d}$ through a sequence of face maps applications $\alpha_{i_{1}}\dotsm \alpha_{i_{k}}q_{d}=q^{b}$ of correct indices. 
    
    %Call this a \emph{delta-chain} or $\delta$-chain and denote by $\deltachain{}$, sometimes indexed.  The canonical naming of a cell $q^{b}$ is thus obtained through an application of a $\delta$-chain to the label $(\executing,\dots,\executing)$ corresponding to $q_{d}$, where each $\alpha_k$ application is as in Definition~\ref{def:canonical-naming}.  
    %Note that one $s_k$ transforms one $\executing$ into $0$, that is, the k-th $\executing$ value from the tuple, and one $t_k$ transforms one $\executing$ into $1$.\uli{Not very clear.  Is it needed?}
  
  
    %The above construction essentially labels the $d$-bulk with lists of Chu-labels, that is, with states from the Chu space on $d$ events.  Any cell $q^{b}$ in a bulk $\bulk{d}$ is reached from the highest cell $q_{d}$ through a sequence of face maps applications $\stmap{i_{1}}\circ\dots\circ\stmap{i_{k}}(q_{d})=q^{b}$ of correct indices. Call this a \emph{delta-chain} or $\delta$-chain and denote by $\deltachain{}$, sometimes indexed.

    %The canonical naming of a cell $q^{b}$ is thus obtained through an application of a $\delta$-chain to the label $(\executing,\dots,\executing)$ corresponding to $q_{d}$, where each $\stmap{k}$ application is as in Definition~\ref{def:canonical-naming}.

    %Note that one $\smap{k}$ transforms one $\executing$ into $0$, that is, the k-th $\executing$ value from the tuple, and one $\tmap{k}$ transforms one $\executing$ into $1$.

    \begin{lemma}
        \label{lemma:chu-labelling-bulk}
        The structure $\bulk{d}$ as defined above is the $d$-dimensional bulk.
    \end{lemma}

    \begin{proof}
        It is trivial to check that $B$ is a precubical set.  $\bulk{d}$ is also non-selflinked, and $\num{ B_d}= \num{\{( \executing,\dotsc, \executing)\}}= 1$, thus by uniqueness up to isomorphism, $\bulk{d}$ is the $d$-dimensional bulk.
    \end{proof}

    The \emph{initial state} $\initbulk{d}$ of the bulk $\bulk{d}$ is the cell named $( 0,\dotsc, 0)$ in the canonical naming.  (any automorphism of $\bulk{d}$ fixes this cell.)  This turns bulks into $\HDA$.

    The $d$-dimensional bulk can be embedded into any bulk of dimension $d'\ge d$ by using the embedding $\embedMorphism^{d'}_{d}:\bulk{d}\hookrightarrow \bulk{d'}$ which maps any cell $( t_1,\dotsc, t_d)$ to $( t_1,\dotsc, t_d, 0,\dotsc, 0)$ in the canonical naming.  It can easily be shown that up to isomorphism, $\embedMorphism^{ d'}_d$ is the \emph{only} $\HDA$ morphism from $\bulk{ d}$ to $\bulk{ d'}$, hence also that there are no $\HDA$ morphisms $\bulk{d}\to \bulk{d'}$ for $d'< d$.

    \begin{definition}[Sculpture]
        \label{def:sculptures}
        A \emph{sculpture} is a HDA $\mathcal{X}$ together with a bulk $\bulk{d}$ and a $\HDA$ embedding morphism, in other words, an injective morphism, $\embedMorphism: X\hookrightarrow \bulk{d}$.
        % for which $\embedMorphism( i)=( 0,\dotsc, 0)$ in the canonical naming.
 
        % \cjlong{There can be different embeddings into different bulks, that is, of different dimensions, all of which can be considered simplistic, in the sense defined below.}

        \emph{morphism} of sculptures $\embedMorphism: \mathcal{X} \hookrightarrow \bulk{ d}$, $\embedMorphism': \mathcal{X}' \hookrightarrow \bulk{ d'}$ is a pair of $\HDA$ morphisms $f: \mathcal{X} \to \mathcal{X}'$, $b: \bulk{ d}\to \bulk{ d'}$ such that the square

        \begin{equation*}
            \xymatrix{%
            \mathcal{X} \ar[d]_{\embedMorphism} \ar[r]_f & \mathcal{X}' \ar[d]^{\embedMorphism'}
            \\ \bulk{d} \ar[r]^b & \bulk{d'}
            }
        \end{equation*}

        commutes, namely, $b\circ \embedMorphism= \embedMorphism'\circ f$.  By the above considerations, this entails that $d'\ge d$ and $b$ is injective, hence also $f$ must be injective.  To sum up, morphisms of sculptures are commuting squares of embeddings. Two sculptures are isomorphic, denoted $\isomorphic$, when the morphism between them has both $f,b$ bijections; thus making the bulks of the same dimension.
    \end{definition}

%    An embedding is, generally, a morphism which is in some sense a isomorphism onto its image.
    
<<<<<<< HEAD
    For the special case of $\mathcal{X} = \mathcal{X}'$ above, we see that any sculpture $\embedMorphism: \mathcal{X} \hookrightarrow \bulk{d}$ can be \emph{over-embedded} into a sculpture $\embedMorphism_d^{d'}\circ \embedMorphism: \mathcal{X} \hookrightarrow \bulk{d'}$ for $d'> d$.  Conversely, any sculpture $\embedMorphism: \mathcal{X} \hookrightarrow \bulk{d}$ admits a \emph{minimal} bulk $\bulk{d_\textup{min}}$ for which $\mathcal{X} \hookrightarrow \bulk{d_\textup{min}}\hookrightarrow \bulk{d}$, that is, such that there is no embedding of $\mathcal{X}$ into $\bulk{d'}$ for any $d'< d_\textup{min}$.  We call such a minimal embedding \emph{simplistic} and $\evdim \mathcal{X} := d_\textup{min}$ the \emph{event dimension} of $\mathcal{X}$; generally, $\evdim \mathcal{X} \ge \dim \mathcal{X}$.
=======
    %For the special case of $\mathcal{X} = \mathcal{X}'$ above, we see that any sculpture $\embedMorphism: \mathcal{X} \hookrightarrow \bulk{d}$ can be \emph{over-embedded} into a sculpture $\embedMorphism_d^{d'}\circ \embedMorphism: \mathcal{X} \hookrightarrow \bulk{d'}$ for $d'> d$.  Conversely, any sculpture $\embedMorphism: \mathcal{X} \hookrightarrow \bulk{d}$ admits a \emph{minimal} bulk $\bulk{d_\textup{min}}$ for which $\mathcal{X} \hookrightarrow \bulk{d_\textup{min}}\hookrightarrow \bulk{d}$, that is, such that there is no embedding of $\mathcal{X}$ into $\bulk{d'}$ for any $d'< d_\textup{min}$.  We call such a minimal embedding \emph{simplistic} and $\evdim \mathcal{X} := d_\textup{min}$ the \emph{event dimension} of $\mathcal{X}$; generally, $\evdim \mathcal{X} \ge \dim \mathcal{X}$.
>>>>>>> 6a85ddbdff57f2fe030ed38810e4ef49c2b1c3cd

    % We say that a sculpture $\sculpture{H}{n}=(H,B_{n},\embedMorphism)$ can be over-complicated to become the sculpture $\sculpture{H}{m}=(H,B_{m},\embedMorphism')$ of higher dimension by taking the $\embedMorphism'=\embedMorphism^{m}_{n}\circ\embedMorphism$. Therefore, we are interested in the minimal bulks, when they exist. We call a sculpture \emph{simplistic} if it cannot be simplified, that is, has a minimal bulk. A sculpture $\sculpture{H}{n}=(H,B_{n},\embedMorphism)$ can be \emph{simplified} when $\exists i\leq n: s_{i}(B_{n})=q_{n-1}$ and $\embedMorphism$ is also an embedding into the smaller bulk, that is, $\embedMorphism:H\rightarrow B_{n-1}$.\cj{Maybe define these notions of Simplistic in the style of Uli.}

    % \begin{proposition}
    %   It is decidable whether a given precubical set $X$ is a sculpture.
    % \end{proposition}

    % \begin{proof}
    %   Let $M= \dim X\cdot \num{ X_{ \dim X}}$.  We claim that if $X$ is a sculpture, then $\evdim X\le M$.  Lemma~\ref{le:thebulk} then gives a procedure to check whether there is an embedding of $X$ into the $M$-dimensional bulk.\uli{Show that $\evdim X\le M$.}
    % \end{proof}


    \begin{remark}
        One precubical set can be seen as two different sculptures, for example, from two different dimensional bulks, in both cases being a simplistic sculpture, meaning it all depends on the embedding morphism. Then this precubical set enters as source object of several sculpture morphisms, as seen in Figure~\ref{fig:asymmetric-conflict-st-hda}.
        
%        \cjlong{We need to redraw these figures once decided.} Because of this we cannot determine from a $\HDA$ alone in which sculpture it enters.

        %Working with history unfoldings is not particularly good either. The interleaving square from Figure~\ref{fig_ex_interleaving_triangle}(left) can be seen as a sculpture from 2, but its history unfolding can be seen as a sculpture from 3 or from 4; we cannot decide which.

%        All the sculptures from Figures~\ref{fig:asymmetric-conflict-st-hda} and~\cite[Figure 5 (right)]{Johansen16STstruct} are simplistic.
    \end{remark}

    %%%% End Definition and notes.
    
    %The mapping ST works like an unfolding since it works with paths; in fact it is more close to the history unfolding of the HDA that it manipulates (cf. Definition 2.19). This is obvious from the example in Figure 5 (right) where the right structure is the unfolding of the left triangle-like HDA. But history-unfolding is h-bisimilar to the original structure, so we could try to check if ST is holds up to h-bisimulation. The example is Figure 4, disregarding the two dotted transitions, also shows two h-bismilar HDA the left being the history-unfolding of the right one, and which are mapped into the same ST-structure. So we could try to show that for HDA that are not h-bisimilar the ST would map them to non-h-bisimilar ST-structures. But this is dismissed by the example of Figure 4, this time considering also the dotted transitions. These two HDA are not h-bisimilar, butthey are mapped to isomorphic ST-structures, hence h-bisimilar.
    
<<<<<<< HEAD
    The major problem with both translations is related to the fact that it is not clear how to identify the events in a higher-dimensional automaton. The best example for this is the fact that the mapping ST destroys the interleaving square; which is due to the fact that the standard method of identifying events in a higher-dimensional automaton by equivalent transitions opposite in a filled square fails for this unfilled square; see Figure \ref{fig:HDA-broken-box} (left). This same issue about events is also the one that causes the problem for the other mapping H where we could not say in the higher-dimensional automaton that was generated whether this was representing two or three events.
=======
    The major problem with both translations is related to the fact that it is not clear how to identify the events in a higher-dimensional automaton. The best example for this is the fact that the mapping ST destroys the interleaving square, which is due to the fact that the standard method of identifying events in a higher-dimensional automaton by equivalent transitions opposite in a filled square fails for this unfilled square; see Figure \ref{fig:HDA-broken-box} (left). This same issue about events is also the one that causes the problem for the other mapping H where we could not say in the higher-dimensional automaton that was generated whether this was representing two or three events.
>>>>>>> 6a85ddbdff57f2fe030ed38810e4ef49c2b1c3cd
    
    However, we show how sculpting allows to identify the events in a higher-dimensional automaton in the same way as ST-structures work with events (we still use the standard intuitive method of seeing events as equivalence classes of transitions opposite in a filled square).
    
    We show that the ST-structures captures precisely the higher-dimensional automaton that can be seen as sculptures. 
    
    %Not all HDA can be sculpted, as shown by Proposition \ct{provide the proposition}.
    
    %Moreover, the sculpting method seems orthogonal to the history-unfolding, as show in theorem \ct{provide theorem}.
    
    %\begin{theorem}[sculpting and unfolding]\
    %    \begin{enumerate}
    %        \item There are HDA which are sculptures but for which their history unfolding is not a sculpture.
    %        \item There are HDA which are not sculptures, but for which their history unfolding is a sculpture.
    %        \item There are HDA which are sculptures, and also their history-unfolding is a sculppture (of a different dimension though).
    %        \item There are HDA which are not sculptures and also their history-unfoldings are not sculptures either.
    %    \end{enumerate}
   % \end{theorem}
    
   % \begin{proof}
   %     \begin{enumerate}
   %         \item This is the example of van Glabbeek \cite[Fig.11]{Glabbeek06HDA}, pictured here in Figure \ref{fig:broken_box}, with the cube with one face missing, and its strange looking unfodling where the corner is split into two.
   %         \item Consider the example from Figure \ref{fig:unfolded triangle}(right) with the triangle where the end state is reached either through one event or through a sequence of two events.
   %         \item Consider example from Figure \ref{fig:unfolded triangle}(left) of the interleaving square.
   %         \item This is the example of the game of the angelic vs. demonic choice from Example \ref{exp:angelic and demonic}, on page \ct{Add page!}, that depends on the speed of the players.
   %     \end{enumerate}
   % \end{proof}
    
   % \ctlong{Need to provide proof.}

   % \begin{itemize}
   %     \item A notion given by Pratt, and attempted to be realized in this thesis.
   %     \item Its relation and placing compared to other models. (ST-structures and HDA)
   %     \item Identifying sculptures (Algorithm that checks if an HDA is a sculpture).
   % \end{itemize}

    %In the cubical set approach to higher dimensional automata an automaton is (possibly infinite) set of cubes of various dimensions. In the Chu space approach one starts instead with a single cube of very large, possibly infinite, dimensions and "sculpts" the desired process by removing unwanted faces. The axes of the starting cube constitute the events, initially a discrete or unstructured set. The removal of states has the effect of structuring the event set. For example sculpting renders two events equivalent, or synchronized, after all states distinguishing them have been removed.

    %The sculpture way of looking at Chu spaces does not reveal the intrinsic symmetry of events and states. An alternative presentation that brings out the symmetry better is a matrix whose rows and columns are indexed by events and states respectively, and whose entries are draw from the set 3={0,1,2}. The column of this matrix constitute the selected faces of the cube.

    % \ctlong{(Use definition from ST-structure to define Sculptures in terms of HDAs)}


    % \ctlong{The category theory relate sculptures. It is important to include this from the CONCUR paper written last year.}


