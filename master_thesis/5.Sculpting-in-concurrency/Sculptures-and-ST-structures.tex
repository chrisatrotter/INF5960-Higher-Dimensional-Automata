\section{Sculptures and ST-structures}
    \label{sec:sculpture-and-st-structure}

    We show that sculptures and ST-structures are isomorphic while also respecting the computation steps. This result resolves the open problems noticed in \cite[Section 3.3]{Johansen16STstruct} that there is no adjoint between the two translations given there between ST-structures and higher-dimensional automata. We are using here the same translations, but add more information that is given by the bulk and the embedding of the sculpture, which allows us to obtain a one-to-one mapping between ST-structures and this class of higher-dimensional automata that sculptures form.
    
    Through the observation from Section~\ref{sec:Chu-spaces-and-ST-structures}, the results in this Section extend to Chu spaces over \textbf{3} as well. Thus, Chu spaces are not enough to capture the concurrency that general higher-dimensional automata can express.
    
    From Section \ref{sec:st-structure-and-hda}, we considered the mapping of $\allST$ into $\allHDA$ which was taken from \cite{Johansen16STstruct}. We showed that for a rooted, connected, and adjacent-closed ST-structure $\ST$ the mapping associates a higher dimensional automaton respecting all precubical identities and is acyclic and non-degenerate. Also, we showed that the order in which the events were picked did not matter.
     
    This mapping can be extended to generate a sculpture, in other words, the dimension of the sculpture and the embedding, thus retaining the information about the events from the ST-structure it encodes.

    \begin{definition}[\allST\ to sculptures]
        \label{def:ST-to-Sculptures} 
        We define a mapping $\stintosculpture$ that for any ST-structure generates an $\HDA$, as well as a bulk and an embedding, thus a sculpture, as follows.  Consider an $\ST=(E,ST)$, with the events linearly ordered as a list $\evlist{E}$. Then $\stintosculpture(\ST)$ returns the $\HDA$ which
  
        \begin{itemize}
            \item has cells $\mathcal{Q}=\{q^{(\mathcal{S},\mathcal{T})}\in \mathcal{Q}_{n} \mid (\mathcal{S},\mathcal{T})\in ST \mbox{ and } \num{(\mathcal{S} \setminus \mathcal{T})}=n\}$;
            \item for any two cells $q^{(\mathcal{S},\mathcal{T})}$ and $q^{(\mathcal{S} \setminus e,\mathcal{T})}$ add the map entry $s_{i}(q^{(\mathcal{S},\mathcal{T})})=q^{(\mathcal{S} \setminus e,\mathcal{T})}$ where $i$ is the index of the event $e$ in the listing $\evlist{E}\!\!\downarrow_{(\mathcal{S} \setminus \mathcal{T})}$;
            \item for any two cells $q^{(\mathcal{S},\mathcal{T})}$ and $q^{(\mathcal{S},\mathcal{T} \cup e)}$ add the map entry $t_{i}(q^{(\mathcal{S},\mathcal{T})})=q^{(\mathcal{S},\mathcal{T} \cup e)}$ where $i$ is the index of the event $e$ in the listing $\evlist{E}\!\!\downarrow_{(\mathcal{S} \setminus \mathcal{T})}$.
        \end{itemize}
        
        $\evlist{E}\!\!\downarrow_{(\mathcal{S} \setminus \mathcal{T})}$ is the listing $\evlist{E}$ restricted to the set $\mathcal{S} \setminus \mathcal{T}$.
 
        Build the bulk $\bulk{n}$, with $n=\num{(E)}$, as in Lemma~\ref{lemma:chu-labelling-bulk} using the canonical naming on the same listing of the events $\evlist{E}$ as used to generate the above $\HDA$.  The embedding $\embedMorphism:\stintosculpture(\ST)\hookrightarrow \bulk{n}$ is defined as $\embedMorphism(q^{(\mathcal{S},\mathcal{T})})=\stintochu(\mathcal{S},\mathcal{T})$ returning the Chu-labelling as in Proposition~\ref{prop:ST-structure-Chu-over-3} on the same listing of events $\evlist{E}$.
    \end{definition}
    
    The mapping $\stintosculpture$ translates a rooted, connected, and adjacent-closed ST-structure $\ST$ into a higher dimensional automaton, respecting all cubical laws. Moreover, it is immaterial which listing of the events is picked in the definition (these results are direct adaptations of results from \cite{Johansen16STstruct}). This mapping preserves isomorphism of ST-structures, and moreover, it does not collapse non-isomorphic ST-structures.

    \begin{proposition}
        \label{prop:ST-to-Sculpture-iso}
         For ST-structures $\ST$ and $\ST'$, $\ST\isomorphic \ST'$ iff\/ $\stintosculpture(\ST)\isomorphic\stintosculpture(\ST')$
    \end{proposition}

    \begin{proof}
         It is clear that $\ST\isomorphic \ST'$ implies $\stintosculpture(\ST)\isomorphic\stintosculpture(\ST')$.  We now show that $\stintosculpture(\ST)\isomorphic\stintosculpture(\ST')$ implies $\ST\isomorphic\ST'$. We show the existence of an isomorphism over the ST-structures, knowing the isomorphism over their translations as sculptures, making use of a fixed listing of events that the map \stintosculpture\ works with, the same listing for both translations, since we know the translation can choose any listing.
        % The rest is tedious details.

        Since the two sculptures $\stintosculpture(\ST)\isomorphic\stintosculpture(\ST')$ are isomorphic it means that the $\HDA$s are embedded in the same bulk, which means we are working with the same set of events $E=E'$ generated by \stintosculpture, modulo a reordering of their listing, which is captured by the second map $b$ between the bulks. We know that $b$ is a bijection on the cells of the bulks, and thus, through the canonical naming scheme, $b$ induces a bijection on $E$ which we call $g$ defined by $g(e_{i})=e'_{i}$ where the index $i$ is important (i.e., $g$ associates the event on position $i$ in the listing $\evlist{E}$ with the event on the same position in the listing $\evlist{E'}$). We use $g$ as the isomorphism that we are looking for between the two ST-structures. 

        We need to show that $g$ preserves configurations, i.e., for some $(\mathcal{S},\mathcal{T})\in\ST$ we prove that $g(\mathcal{S},\mathcal{T}) \in \ST'$. We make use of the commuting square property of the sculpture morphisms. For the ST-configuration there exists the corresponding cell $q^{(\mathcal{S}, \mathcal{T})}$ generated by the \stintosculpture, which in turn is embedded in the bulk as $\stintochu(\mathcal{S},\mathcal{T})$. At the same time the HDA morphism associates $f(q^{(\mathcal{S},\mathcal{T})})=q^{(\mathcal{S}',\mathcal{T}')}$ to some cell in $\stintosculpture(\ST')$ which in turn comes from some ST-configuration $(\mathcal{S}',\mathcal{T}')$. This cell is embedded into the bulk as $\stintochu(\mathcal{S}',\mathcal{T}')$ which is equated to $\stintochu(\mathcal{S},\mathcal{T})=b(\stintochu(\mathcal{S},\mathcal{T}))$ by the bulk isomorphism, meaning that respective events in the two listings receive the same values. Now, going through the translation between Chu states and ST-configurations we can see how $g(\mathcal{S},\mathcal{T})=(g(\mathcal{S}),g(\mathcal{T}))=(\mathcal{S}',\mathcal{T}')\in\ST'$.

        \begin{itemize}
            \item If $e_{i} \not\in \mathcal{S}$ then $\stintochu(\mathcal{S},\mathcal{T})(e_{i})=0$, which means that also $g(e_{i})=e'_{i}$ takes the same value in $\stintochu(\mathcal{S}',\mathcal{T}')(g(e_{i}))=0$. This means that $g(e_{i})\not\in \mathcal{S}'=g(\mathcal{S})$ because of injectivity.
            \item If $e_{i}\in \mathcal{S}$ and $e_{i} \not\in \mathcal{T}$ then $\stintochu(\mathcal{S},\mathcal{T})(e_{i})=\executing$, which means that also $g(e_{i})=e'_{i}$ takes the same value in $\stintochu(\mathcal{S}',\mathcal{T}')(g(e_{i}))=\executing$. This means that $g(e_{i})\in g(\mathcal{S})$ and $g(e_{i}) \not\in g(\mathcal{T})$ because of injectivity on $\mathcal{S}$.
            \item If $e_{i}\in \mathcal{T} \subseteq \mathcal{S}$ then $\stintochu(\mathcal{S},\mathcal{T})(e_{i})=1$ which means that also $g(e_{i})=e'_{i}$ takes the same value in $\stintochu(\mathcal{S}',\mathcal{T}')(g(e_{i}))=1$. This means that $g(e_{i}) \in g(\mathcal{T}) \subseteq g(\mathcal{S})$.
        \end{itemize}
    
        This proves that $(g(\mathcal{S}),g(\mathcal{T}))\in\ST'$.
    
    \end{proof}

    \begin{definition}[Sculptures to $\allST$]
        \label{def:sculptures-to-ST}
        Define a mapping $\sculpintost$, which for a sculpture $\sculpture{\mathcal{H}}{n}=(\mathcal{H},\bulk{n},\embedMorphism)$ associates the ST-structure $\sculpintost(\sculpture{\mathcal{H}}{n})$ as follows. Take a linearly ordered set $\evlist{E}$ (of events) of cardinality as the dimension of the bulk $n=\dim\bulk{n}$. The ST-configurations of $\sculpintost(\sculpture{\mathcal{H}}{n})$ are obtained from the cells of $H$ as $\sts=\{\chuintost(\embedMorphism(q)) \mid q\in \mathcal{H}\}$.
    
    \end{definition}

    Intuitively, since we have the bulk we can work with the canonical naming based on a fixed listing of events. Using Proposition~\ref{prop:ST-structure-Chu-over-3} we can associate an ST-configuration to each cell of the $\HDA$ by going through the embedding to the corresponding cell in the bulk. It is clear that $\sculpintost(\sculpture{\mathcal{H}}{n})$ is rooted, connected and closed under single events, that is, regular.

    The following results show a one-to-one correspondence between the ST-structures and sculptures.

    \begin{proposition}
        \label{prop:ST-to-Sculpture}
        For an arbitrary ST-structure $\ST$ we have
        \[
            \sculpintost(\stintosculpture(\ST))\isomorphic \ST.
        \]
    \end{proposition}

    \begin{proof}
        To prove the isomorphism between the right-hand $\ST=(E,\sts)$ and the left-hand $\sculpintost(\stintosculpture(\ST))$ we need to show that $\exists f: E \parto E'$ such that $f$ preserves ST-configurations.  
        % and is locally injective and total, 
        % i.e., for all $( S, T)\in \sts$ we show that $f( S, T):=( f( S), f( T))\in \sts'$, with $\sts'$ the ST-configurations generated on the left-hand side.
        %  and for all $( S, T)\in \sts$, the restriction $f\rest S$ is injective and total). 
        We will consider $E'$ to be the events produced by the left-hand side of the isomorphism, precisely, by $\sculpintost(\stintosculpture(\ST))$.
    
        The application of the mapping $\stintosculpture$ 
        % provided in Definition \ref{def_stintosculptures} which 
        generates an $\HDA$, as well as a bulk and an embedding, considering the events of $\ST$ to be linearly ordered as a list $\evlist{E}$. $\stintosculpture$ builds cells $\mathcal{Q}=\{q^{(\mathcal{S},\mathcal{T})}\in \mathcal{Q}_{n} \mid (\mathcal{S},\mathcal{T})\in\sts \mbox{ and } \num{(\mathcal{S} \setminus \mathcal{T})}=n\}$ and the embedding $\embedMorphism(q^{(\mathcal{S},\mathcal{T})})=\stintochu(\mathcal{S},\mathcal{T})$ into the bulk $\bulk{\num E}$; thus the sculpture $\sculpture{\mathcal{H}}{n}=(\mathcal{Q}, \bulk{\num E}, \embedMorphism(q^{(\mathcal{S},\mathcal{T})}))$. The embedding $\embedMorphism$ returns for each cell $q^{(\mathcal{S}, \mathcal{T})}$ the Chu-labelling as in Proposition~\ref{prop:ST-structure-Chu-over-3} on the same listing of events $\evlist{E}$.
   
        Thus, the ST-structure produced on the left-hand side by $\sculpintost$ applied to the above sculpture $\sculpture{\mathcal{H}}{s}(\ST)=\sculpture{\mathcal{H}}{n}$ assumes, without loss of generality, to have the same listing of events $\evlist{E}$ as before.
        % , i.e., as produced by \sculpintost\ according to Definition~\ref{def_sculptures_to_ST}.
        % 
        The ST-configurations of $\sculpintost(\sculpture{\mathcal{H}}{n})$ are obtained from the cells of $\sculpture{\mathcal{H}}{n}$ as $\sts' = \{\chuintost(\embedMorphism(q^{(\mathcal{S},\mathcal{T})})) \mid q^{(\mathcal{S},\mathcal{T})} \in \sculpture{\mathcal{H}}{n}\}$, see Definition \ref{def:sculptures-to-ST}. 

        Since $E'$ is the same set of events $E$, we take the isomorphism map $f: E \rightarrow E$ to be identity function on $E$.
        %, i.e., $\forall e\in E: f(e)=e$.
        % We show that the isomorphism properties hold. 
        We check that $f$ preserves the ST-configurations, such that $f(\mathcal{S}, \mathcal{T}):=( f(\mathcal{S}), f(\mathcal{T})) = (\mathcal{S}, \mathcal{T}) \in \sts'$. From this it must be that $\exists q^{(\mathcal{S},\mathcal{T})}\in\sculpture{\mathcal{H}}{n}: \chuintost(\embedMorphism(q^{(\mathcal{S},\mathcal{T})}) = (\mathcal{S},\mathcal{T})$ by Definition \ref{def:sculptures-to-ST}. Further, from Definition~\ref{def:ST-to-Sculptures} we know that $\chuintost(\embedMorphism(q^{(\mathcal{S},\mathcal{T})}) = \chuintost(\stintochu(\mathcal{S},\mathcal{T}))$. Finally, by Proposition \ref{prop:ST-structure-Chu-over-3} we have the expected result $\chuintost(\embedMorphism(q^{(\mathcal{S},\mathcal{T})}) = \chuintost(\stintochu(\mathcal{S},\mathcal{T})) = (\mathcal{S},\mathcal{T})$.

        It is easy to see that $f$, as identity function, is locally injective and total.
    \end{proof}

    \begin{proposition}
        \label{prop:Sculpture-to-ST}
        For any sculpture $\sculpture{\mathcal{H}}{n}=(\mathcal{H},\bulk{n},\embedMorphism)$ we have
        \[
            \stintosculpture(\sculpintost(\sculpture{\mathcal{H}}{n})) \isomorphic \sculpture{\mathcal{H}}{n}.
        \]
    \end{proposition}

    \begin{proof}
        To prove the isomorphism between the right-hand $\mathcal{H}^{n}=(\mathcal{H},B^{n}, em)$ and the left-hand $\stintosculpture(\sculpintost(\mathcal{H}^{n}))$ we need to show that $\exists f: \mathcal{H} \rightarrow \mathcal{H}'$ and $\exists b:B^{n} \rightarrow B^{n}$ such that $f,b$ are bijective and the square commutes, namely, $b \circ em = em' \circ f$.
    
        \begin{equation*}
            \xymatrix{%
            \mathcal{H} \ar[d]_{\embedMorphism} \ar[r]_f & \mathcal{H}' \ar[d]^{\embedMorphism'}
            \\ \bulk{n} \ar[r]^b & \bulk{n'}
            }
        \end{equation*}

        First, we consider the mapping $\sculpintost$ provided in Definition \ref{def:sculptures-to-ST} which generates an $ST$ for a sculpture, as follows. Take a linearly ordered set $\evlist{E}$ (of events) of cardinality as the dimension of the bulk, that is, $|E| = n = \dim \bulk{n}.$ The ST-configurations of $\sculpintost(\mathcal{H}^{n})$ are obtained from the cells of $\mathcal{H}$, such that $\forall q \in \mathcal{H} :\sculpintost(q)=STChu(\embedMorphism(q))=(\mathcal{S},\mathcal{T})^{q}$. This is the ST-structure produced on the left-hand side by $\sculpintost$ applied to the sculpture $\mathcal{H}^{n}$. 

        We will now consider the mapping $\stintosculpture$ provided in Definition \ref{def:ST-to-Sculptures} which generates a $\HDA$ for any ST-structure, as well as a bulk and an embedding, thus a sculpture, as follows. $\stintosculpture$ requires the events of the ST-structure to be linearly ordered. We take the same order produced above by $\sculpintost$.
    
        \begin{enumerate}
            \item Build the bulk $B^{n'}$, with $n'=\num E=n$, as in Lemma \ref{lemma:chu-labelling-bulk} using the canonical naming on the listing of the events $\evlist{E}$. This is thus the same bulk $\bulk{n}$ from $\mathcal{H}^{n}$. Thus we can take the $b$ part of the sculptures morphism to be the identity function which is thus a morphism between \HDA. This is also a bijection.
        
            \item The HDA $\stintosculpture(\sculpintost(\mathcal{H}^{n}))$ has cells $\mathcal{Q}=\{p^{(\mathcal{S},\mathcal{T})} \in \mathcal{Q}_{n} \mid (\mathcal{S},\mathcal{T}) \in \sculpintost(\mathcal{H}^{n})$ and $\num{(\mathcal{S} \setminus \mathcal{T})}=n\}$. Since each ST-configuration corresponds to some $q \in \mathcal{H}^{n}$ we thus construct one cell $p^{(\mathcal{S},\mathcal{T})^{q}}$ for each cell $q$; which according to Definition~\ref{def:sculptures-to-ST} is built using the embedding, in other words, $p^{(\mathcal{S}, \mathcal{T})^{q}} = p^{\chuintost(\embedMorphism(q))}$.
            
            \item The embedding $\embedMorphism':\stintosculpture(\sculpintost(\mathcal{H}^{n})) \hookrightarrow B^{n'}$ is defined as $\embedMorphism'(p^{(\mathcal{S},\mathcal{T})}) = Chu(\mathcal{S},\mathcal{T})$ returning the Chu-labelling as in Proposition \ref{prop:ST-structure-Chu-over-3} on the same listing of events $\evlist{E}$.
            
            We thus take $f: \mathcal{H} \rightarrow \stintosculpture(\sculpintost(\sculpture{\mathcal{H}}{n}))$ to be defined as $f(q)=p^{\chuintost(\embedMorphism(q))}$ and $f^{-1}(p^{\chuintost(\embedMorphism(q))})=q$; thus obtaining a bijection between the cells of the respective $\HDA$s.
            
            We show that $f$ is a morphism of $\HDA$s. We thus show that $f$ commutes with the face maps, such that for any $q$ we show $s_{i}(f(q))=f(s_{i}(q))$. We have $s_{i}(f(q)) = s_{i}(p^{\chuintost(\embedMorphism(q))})=s_{i}(p^{(\mathcal{S},\mathcal{T})})$ if we call $\chuintost(\embedMorphism(q))=(\mathcal{S},\mathcal{T})$. Definition \ref{def:ST-to-Sculptures} relates this $i$-th map to a cell made from the ST-configuration $(\mathcal{S} \setminus e,\mathcal{T})$ with $e$ being the $i$-th event in the listing $\evlist{E}\restrictedToSet{\mathcal{S}\setminus \mathcal{T}}$. Since the bulk uses the same listing, we thus have this ST-configuration obtained as $\chuintost(\embedMorphism(s_{i}(q)))$. On the right-hand side of the equality we have the same by definition $f(s_{i}(q))=p^{\chuintost(\embedMorphism(s_{i}(q)))}$.

            We show that the sculptures morphism square commutes. The fact that $f,b$ are bijections finishes the proof, which proves the isomorphism property. We show that $\embedMorphism(q)=\embedMorphism'(f(q))$ by working with the right-hand side: $\embedMorphism'(f(q))=\embedMorphism'(p^{\chuintost(\embedMorphism(q))})=\stintochu(\chuintost(\embedMorphism(q)))=\embedMorphism(q)$.
        \end{enumerate}    
    \end{proof}
    
    We can also understand $\sculpintost$ as labelling every cell of the sculpture with a ST-configuration, or equivalently (because of Proposition~\ref{prop:ST-structure-Chu-over-3}) with a Chu-3 state, which in the terminology of Lemma~\ref{lemma:chu-labelling-bulk} we call this a Chu-listing or Chu-label. Corollary~\ref{cor:unqiue-chu-label-in-bulk} says that this labelling is unique. Thus, in sculptures we have a one-to-one correspondence between HDA states, ST-configurations, and Chu-3 states.

    \begin{corollary}\label{cor:unqiue-chu-label-in-bulk}
        In a bulk every cell has a unique label (either as a ST-configuration or as a Chu-label representation). Thus, there are no two cells of the bulk with the same label.
    \end{corollary}

    \begin{proof}
        This is easy to see from the proofs above.
        % \cj{Do we need more detailed proof here?}
    \end{proof}