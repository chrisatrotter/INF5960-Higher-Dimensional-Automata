In this chapter we present non-interleaving models of concurrency that generalize the independence of two, or more, events. We present a geometric model of concurrency which is an algebraic structure able to capture the main characteristic of \emph{both} interleaving and non-interleaving models. We show both a state-based model, higher-dimensional automata, and an event-based model, ST-structures, capable of distinguishing an arbitrary number of events. Also, we present Chu spaces as a model able to present state-based and event-based models symmetrically, such that there is a way to present the event-state duality symmetrically. 

The models mentioned so far have been based on interleaving computation steps to capture all the possible behaviours of a concurrent system. Recall the interleaving square, in Figure \ref{fig:transition-system}, where we have four states $s_{1}$, $s_{2}$, $s_{3}$ and $s_{4}$ and four transitions, where two of the transitions are labeled $a$ and the other two transitions are labeled $b$. We see that these two actions $a$ and $b$ can be executed in either order, ending in the same state each way, such that $a$ and $b$ are mutually exclusive. By design, interleaving models depend on the notion of \emph{atomic} actions, that is, actions that are indivisible.

The assumption of atomic actions is reasonable for events that are mutually exclusive, meaning they cannot occur simultaneously. For example, a vending machine able to provide a single item at a time, like a snack or a beverage. If we want two items, such as a chocolate and a beverage, from the vending machine, then either the chocolate will be released followed by the beverage or first the beverage followed by the chocolate. However, for events that are truly independent, such as a vending machine able to provide two, or more, items at a time, then their order of occurrence would be rendered irrelevant $-$ two items would be released at the same time from the vending machine. Models that only allow interleaving are unable to distinguish these two vending machine. Instead, we will consider the non-interleaving models of concurrency, such as \emph{asynchronous transition systems} \cite{winskel95modelsCategory} and \emph{higher-dimensional automata} \cite{pratt91hda}, \emph{ST-structures} \cite{Johansen16STstruct} and \emph{Chu spaces} \cite{gupta94phd_Chu}, which are able to distinguishing between these two vending machines. Thus, we allow receiving multiple items after each other, or all at the same time. Going further, we may even have a combination of the alternatives.

%This notion is reasonable for events that actually are mutually exclusive, such as two people wanting to pass through a single door, at the same time, where only one of them can pass through at a time. But for events that are truly independent, such as two, or more, people wanting to pass through a single door, at the same time, where the door is large enough to allow them to pass through without bumping into each other. Their order of occurrence would be rendered irrelevant. If we consider the interleaving approach, then we are unable to distinguish these two scenarios. However, if we consider the non-interleaving models of concurrency, such as \emph{asynchronous transition systems} \cite{winskel95modelsCategory} and \emph{higher-dimensional automata} \cite{pratt91hda}, then we are able to distinguish such scenarios.

%In the former, we are unable to distinguish between the execution of two non-interleaving concurrent actions and of two mutually exclusive actions as they both are modeled by interleaving. While in the latter, we are able to distinguish between these executions if we consider non-interleaving models of concurrency such as \emph{asynchronous transition systems} \cite{winskel95modelsCategory} and \emph{higher-dimensional automata} \cite{pratt91hda}.

For didactic and historical reasons, we will review asynchronous transition systems where we introduce an independence relation between the occurrence of events. The independence relation is able to express the mutual exclusion and non-interleaving of two events. Asynchronous transition systems may be considered a bridge between transition systems, used up until now, and the higher-dimensional automata, introduced in Section \ref{sec:higher-dimensional-automata}. Higher-dimensional automata are expressive and can encode the independence of an arbitrary number of events, rather than only pairs of events.

Higher-dimensional automata are \emph{algebraic structures} which can be interpreted geometrically, roughly as topological spaces with a sense of direction to incorporate a notion of irreversible time. A topological space with a sense of direction makes it possible to determine the order of an execution. Intuitively, a higher-dimensional automaton is an automaton with nicely incorporated squares, cubes and higher-dimensional cubes, which represent the independence of events. A main characteristic of higher-dimensional automata is to be able to capture what happens \emph{during} an execution. ST-structure is the event-based counterpart of higher-dimensional automata which is able to precisely identify events, and also capture what happens during an execution. As shown by Johansen in \cite{Johansen16STstruct}, higher-dimensional automata cannot precisely identify events. For example, the asymmetric conflict shown in \cite[Figure 5]{Johansen16STstruct} and presented in Section \ref{sec:st-structure-and-hda}. However, we may faithfully represent the asymmetric conflict as a ST-structure.

%Lastly, we will review Chu spaces since it is known that configuration structures correspond to the Chu spaces over 2. The extension with "\emph{duration}" that ST-structures represent in turn corresponds to the extension of Chu spaces to being defined over 3 \ct{Not clear what 2 and 3 refers too!}, since the structure 3 is the extension of 2, which is the simple "start < terminate", to the "start < during < terminate". We will attempt to look at the Chu spaces as isomorphic to ST-structures through the results by Johansen in \cite{Johansen16STstruct}.




%The naming precubical identity is most commonly used in algebraic topology \cite{Goubault95PhDThesis, Fajstrup16DirectedAlgebraicTopologyConcurrency, Kahl2013TheHG}. While in computer science, the precubical identity is called the \emph{cubical law} \cite{Glabbeek06HDA, Johansen16STstruct}.



%Second \ct{Why am I reading about HDAs?}, we will review higher-dimensional automata, a model where \emph{precubical sets} that encode the independence of events by \emph{mappings}. A precubical \ct{why precubcial sets?} set will consist of sets of $n$-dimensional cubes \cite{Fajstrup16DirectedAlgebraicTopologyConcurrency} together \ct{what is this together? Why together?} with their faces: each $n$-dimensional cube has $2n$ faces, in other words a front and a back face in each direction $i$ with $0 \leq i < n$. Mappings are family of functions that identify these faces. For mappings that satisfy the \emph{precubical identity} we may consider them independent. The precubical identity \ct{Id-relation? As category theory? What is this?} is what Pratt considers as the intuitive notion of "\emph{filling in holes}" of a $n$-dimensional cube \cite[Section 2]{pratt91hda}. This notion is shown in both Figure \ref{fig:precubical-set-interleaving-square} (right) and Figure \ref{fig:HDA-filled-interleaving-square}, where the interior of the interleaving square is filled. Higher-dimensional automata \ct{What is the connection between precubical sets and HDAs?} are considered an extension of both transition systems and asynchronous transition systems. With higher-dimensional automata we can distinguish between the execution of $n$ non-interleaving concurrent actions and of $n$ mutually exclusive actions. 

%Interestingly, higher-dimensional automata are \emph{algebraic structures} which can be interpreted roughly as \emph{topological spaces}. However, as pointed out by Fahrenberg in \cite{Fahrenberg05PhD}, there is a problem with the translation from higher-dimensional automata to topological spaces: one looses all information concerning precedence of events. In order to make this connection precise, it turns out that topological spaces are not exactly the right notion for our purpose. One needs to use a \emph{directed variant}, i.e., to incorporate a notion of irreversible time. Preserving the precedence of events and other problems are addressed in \emph{directed topology}, where the objects of study are topological spaces equipped with precedence information \cite{Fahrenberg05PhD, Goubault92homologyof, Fajstrup16DirectedAlgebraicTopologyConcurrency}. \ct{Why? Imporatance?}

%Third, we will review ST-structures, an event-based counterpart of higher-dimensional automata, which is able to identify the precedence of events. ST-structures extend configuration structures \ct{what are configuration structure?!} with the notion of "\emph{during}" as opposed to only talking about "\emph{what happens after}" an event. Therefore, in ST-structures we can see when an event, or several concurrent ones, are currently being executed, as well as when the event has terminated its execution. In \cite{Johansen16STstruct}, Johansen explains that ST-structure capture a main characteristic of higher dimensional automata. More specifically, ST-structures have the possibility to look at the currently executing events. 

%opposed to the standard configuration structures; this is the power to look at the currently executing concurrent events, compared to not only observing their termination.

%Lastly, we will review Chu spaces since it is known that configuration structures correspond to the Chu spaces over 2. The extension with "\emph{duration}" that ST-structures represent in turn corresponds to the extension of Chu spaces to being defined over 3 \ct{Not clear what 2 and 3 refers too!}, since the structure 3 is the extension of 2, which is the simple "start < terminate", to the "start < during < terminate". We will attempt to look at the Chu spaces as isomorphic to ST-structures through the results by Johansen in \cite{Johansen16STstruct}.

 %First, we will review asynchronous transition systems, where we introduce an independence relation between the occurrence of events. The independence relation is able to express the mutual exclusion and non-interleaving of up to two events. Second, we will review higher-dimensional automata, which are equipped with \emph{precubical sets} that encode the independence of events by \emph{mappings}. A precubical set will consists of sets of $n$-dimensional cubes \cite{Fajstrup16DirectedAlgebraicTopologyConcurrency} together with their faces: each $n$-dimensional cube has $2n$ faces, in other words a front and a back face in each direction $i$ with $0 \leq i < n$. Mappings are family of functions that
 
 %A precubical set will consists of sets of $n$-dimensional cubes for each $n \in \mathbb{N}$, together with their faces: each $n$-dimensional cube has $2n$ faces, i.e., a front and a back face in each direction $i$ with $0 \leq i < n$. Mappings are family of functions that identify these faces. For mappings that satisfy the \emph{precubical identity}, we may consider them independent or as referred to by Pratt as the geometric notion of "filling in holes" \cite[Section 2]{pratt91hda}. This geometric notion is shown in Figure \ref{fig:precubical-set-interleaving-square} (right), where the interior of the interleaving square is filled. The precubical identity is a condition that allows us to generate a back face by knowing its corresponding front face, but also generate the front by knowing its corresponding back face. Precubical sets are well known in the literature of directed algebraic topology \cite{Fahrenberg05PhD, Goubault95PhDThesis, Fajstrup16DirectedAlgebraicTopologyConcurrency}.

%Higher-dimensional automata are considered an extension of both transition systems and asynchronous transition systems. With higher-dimensional automata we can distinguish between the execution of $n$ non-interleaving concurrent actions and of $n$ mutually exclusive actions.

%Lastly, we will introduce ST-structures, which is the event-based counterpart of higher-dimensional automata. Followed by Chu spaces, which is a model that was proposed in response to the event-state duality described by Pratt \cite{Pratt02eventStateDuality}. Leading to the notion of sculpting as a method to identify events in higher-dimensional automata.

%These models have been shown to be generalizations of interleaving models of concurrency. Let us first generalize transition systems by asynchronous transition system, then further generalize these to some of the other non-interleaving models of concurrency.

%From a topological perspective, we may consider the above models the same as \emph{asynchronous semantics}\cite[Section 3.3]{Fajstrup16DirectedAlgebraicTopologyConcurrency} and \emph{precubical sets}\cite[Section 3.4]{Fajstrup16DirectedAlgebraicTopologyConcurrency}, respectively.

%These models have been shown to be generalizations of interleaving models of concurrency. Let us first generalize transition systems by asynchronous transition system, then further generalize these to some of the other non-interleaving models of concurrency.

%First, we will consider asynchronous transition systems which can be seen as similar to asynchronous semantics where we introduce an independence relation between the occurrence of events. Furthermore, we will generalize asynchronous transition systems to higher-dimensional automata. Higher-dimensional automata are equipped with precubical sets which is what generalizes the model such that it can generalize for $n$ events. Lastly, we will introduce \emph{ST-structures}, which is the event-based counterpart of higher-dimensional automata, and \emph{Chu spaces}, which is a model that was created as a response to the event-state duality described by Pratt \cite{Pratt02eventStateDuality}.

%We see that the two instructions $a$ and $b$ can be executed in either order, ending in the same state each way, such that $a$ and $b$ are mutually exclusive. By design, we have that interleaving models depend on the notion of \emph{atomic} actions which are considered to be \emph{indivisible}. This notion is reasonable for events that actually are mutually exclusive, such as two people wanting to pass through a single door, at the same time, where only one of them can pass through at a time. But for events that are truly independent, such as two people wanting to pass through a single door, at the same time, where the door is large enough to allow both of them to pass through. Their order of occurrence would be rendered  irrelevant. When considering interleaving models, we are unable to distinguish between the execution of two truly concurrent actions and of two mutually exclusive actions as they are both modeled by interleaving. While with non-interleaving models, we are able to distinguish between these executions by considering non-interleaving models of concurrency such as \emph{asynchronous semantics}\cite[Section 3.3]{Fajstrup16DirectedAlgebraicTopologyConcurrency} and \emph{precubical sets}\cite[Section 3.4]{Fajstrup16DirectedAlgebraicTopologyConcurrency} which have shown to be generalizations of interleaving models of concurrency. First, we will consider asynchronous transition systems which... bla bla bla \ct{need to add something here}. Secondly, we will generalize further asynchronous transition systems to higher-dimensional automata which may be considered precubical sets. Lastly, we will introduce \emph{ST-structures}, which is an event-based counterpart to higher-dimensional automata, and \emph{Chu spaces}, which was a model design as a response to the event-state duality described by Pratt \cite{Pratt02eventStateDuality}.

%We see that the two instructions $a$ and $b$ can be executed in either order, ending in the same state each way, such that $a$ and $b$ are mutually exclusive. By design, we have that interleaving models depend on the notion of \emph{atomic} actions which are considered to be \emph{indivisible}. This notion is reasonable for events that actually are mutually exclusive, such as two people wanting to pass through a single door, at the same time, where only one of them can pass through at a time. But for events that are truly independent, such as two people wanting to pass through a single door, at the same time, where the door is large enough to allow both of them to pass through. Their order of occurrence would be rendered  irrelevant. In the former, we are unable to distinguish between the execution of two truly concurrent actions and of two mutually exclusive actions as there both modeled by interleaving. While the latter, we are able to distinguish between these executions by considering non-interleaving models of concurrency such as \emph{asynchronous semantics}\cite[Section 3.3]{Fajstrup16DirectedAlgebraicTopologyConcurrency} and \emph{precubical sets}\cite[Section 3.4]{Fajstrup16DirectedAlgebraicTopologyConcurrency} which have shown to be generalizations of interleaving models of concurrency. First, we will consider asynchronous transition systems which... bla bla bla \ct{need to add something here}. Secondly, we will generalize further asynchronous transition systems to higher-dimensional automata which may be considered precubical sets. Lastly, we will introduce \emph{ST-structures}, which is an event-based counterpart to higher-dimensional automata, and \emph{Chu spaces}, which was a model design as a response to the event-state duality described by Pratt \cite{Pratt02eventStateDuality}.


%Further, we will review a model of concurrency based on geometry which is considered the most common model for non-interleaving models of concurrency.
% such as two people wanting to pass through a single door that is large enough to fit both of them when passing through the door in the same second. 