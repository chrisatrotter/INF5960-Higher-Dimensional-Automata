In this chapter we present non-interleaving models of concurrency that generalize the independence of two, or more, events. We present a geometric model of concurrency which is an algebraic structure able to capture the main characteristic of \emph{both} interleaving and non-interleaving models. We show both a state-based model, higher-dimensional automata, and an event-based model, ST-structures, capable of distinguishing an arbitrary number of events. Also, we present Chu spaces as a model able to present state-based and event-based models symmetrically, such that there is a way to present the event-state duality symmetrically. 

The models mentioned so far have been based on interleaving computation steps to capture all the possible behaviours of a concurrent system. Recall the interleaving square, in Figure \ref{fig:transition-system}, where we have four states $s_{1}$, $s_{2}$, $s_{3}$ and $s_{4}$ and four transitions, where two of the transitions are labeled $a$ and the other two transitions are labeled $b$. We see that these two actions $a$ and $b$ can be executed in either order, ending in the same state each way, such that $a$ and $b$ are mutually exclusive. By design, interleaving models depend on the notion of \emph{atomic} actions, that is, actions that are indivisible.

The assumption of atomic actions is reasonable for events that are mutually exclusive, meaning they cannot occur simultaneously. For example, a vending machine able to provide a single item at a time, like a snack or a beverage. If we want two items, such as a chocolate and a beverage, from the vending machine, then either the chocolate will be released followed by the beverage or first the beverage followed by the chocolate. However, for events that are truly independent, such as a vending machine able to provide two, or more, items at a time, then their order of occurrence would be rendered irrelevant $-$ two items would be released at the same time from the vending machine. Models that only allow interleaving are unable to distinguish these two vending machine. Instead, we will consider the non-interleaving models of concurrency, such as \emph{asynchronous transition systems} \cite{winskel95modelsCategory} and \emph{higher-dimensional automata} \cite{pratt91hda}, \emph{ST-structures} \cite{Johansen16STstruct} and \emph{Chu spaces} \cite{gupta94phd_Chu}, which are able to distinguishing between these two vending machines. Thus, we allow receiving multiple items after each other, or all at the same time. Going further, we may even have a combination of the alternatives.

For didactic and historical reasons, we will review asynchronous transition systems where we introduce an independence relation between the occurrence of events. The independence relation is able to express the mutual exclusion and non-interleaving of two events. Asynchronous transition systems may be considered a bridge between transition systems, used up until now, and the higher-dimensional automata, introduced in Section \ref{sec:higher-dimensional-automata}. Higher-dimensional automata are expressive and can encode the independence of an arbitrary number of events, rather than only pairs of events.

Higher-dimensional automata are \emph{algebraic structures} which can be interpreted geometrically, roughly as topological spaces with a sense of direction to incorporate a notion of irreversible time. A topological space with a sense of direction makes it possible to determine the order of an execution. Intuitively, a higher-dimensional automaton is an automaton with nicely incorporated squares, cubes and higher-dimensional cubes, which represent the independence of events. A main characteristic of higher-dimensional automata is to be able to capture what happens \emph{during} an execution. ST-structure is the event-based counterpart of higher-dimensional automata which is able to precisely identify events, and also capture what happens during an execution. As shown by Johansen in \cite{Johansen16STstruct}, higher-dimensional automata cannot precisely identify events. For example, the asymmetric conflict shown in \cite[Figure 5]{Johansen16STstruct} and presented in Section \ref{sec:st-structure-and-hda}. However, we may faithfully represent the asymmetric conflict as a ST-structure.