\section{Summary}

As presented Asynchronous transition systems are able to deal directly with concurrency, however, the model is only able to distinguish two events. The geometric model, higher-dimensional automata, is a non-interleaving model with an algebraic structure able to distinguish an arbitrary number of events. Higher-dimensional automata are useful and general and able to capture the behaviour of other interleaving and non-interleaving models \cite{Goubault18RelationshipsModelsForConcurrency}, such as transition systems and asynchronous transition systems \cite{Fajstrup16DirectedAlgebraicTopologyConcurrency}. However, higher-dimensional automata cannot precisely identify events, such as the asymmetric conflict shown in \cite[Figure 5]{Johansen16STstruct}. We will present identifying events in higher-dimensional automata with ST-structures in the next chapter. We will also use Chu spaces as a way to capture the event-state duality of higher-dimensional automata and ST-structures. Chu spaces will provide a way to reconcile the relations of ST-structures and higher-dimensional automata. 