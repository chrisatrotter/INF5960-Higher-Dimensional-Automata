\section{The organisation of this thesis}
In this section, we briefly present the organisation of the thesis with an overview of each chapter.

\textbf{Chapter 2} presents transition systems and event structures. Transition systems are interleaving models of concurrency and can be interpreted, geometrically, as one-dimensional space consisting of edges (its transitions) meeting and branching at vertices (its states). Transition systems provide the foundation to understand the generalization of the geometric model of concurrency. Also, transition systems are automata and considered one side of a duality, where schedules, or events, is the other. Event structures are schedules which provide the necessary background for understanding ST-structures and provide the intuition of the state-event duality presented in Chapter \ref{chap:Relationship with other models of true concurrency}. 


%Introduces one of the most traditional models of concurrency known as transition systems. We consider the transition system as a one-dimensional space consisting of edges (its transitions) meeting and branching at vertices (its states). It will provide us with the foundation to understand the generalization of the geometric model of concurrency. Also providing a way to present the traditional models of concurrency in terms of non-interleaving models of concurrency, specifically the geometric model of concurrency.

\textbf{Chapter 3} presents the non-interleaving models of concurrency, specifically, asynchronous transition systems, higher-dimensional automata, ST-structures and Chu spaces. We consider asynchronous transition systems to be the bridge between transition systems and higher-dimensional automata. Asynchronous transition systems are seen, geometrically, as a two-dimensional space extending the transition systems by considering surfaces (its independence relations) between edges (its transitions) and vertices (its states). Higher-dimensional automata are a generalization of both transition systems and asynchronous transition systems. We consider higher-dimensional automata as an $n$-dimensional space consisting of points, segments, squares, cubes, and higher-dimensional cubes. ST-structures are the event-based counterpart of higher-dimensional automata capable of capturing a main characteristic of higher-dimensional automata, that is, to be able to see what happens during a concurrent execution. Chu spaces are considered a response to the state-event duality problem and introduce the notion of "\emph{sculpting}", which we present in Chapter \ref{chap:sculpting-in-concurrency}.

%Introduces asynchronous transition systems which are the 2-dimensional case of the geometric model of concurrency. Showing that this model is a generalization of transition systems able to distinguish mutual exclusion from non-interleaving concurrency, but only for two concurrent processes. Leading to Higher dimensional automata which is the n-dimensional case that is a generalization of both ordinary transition systems and asynchronous transition systems. We also introduce the ST-structures which is the event-based counterpart to HDA. Finally, we present Chu spaces which are seen to be a response to the state-event duality problem. Chu spaces also first introduce the notion of Sculpting which will be introduced in Chapter \ref{chap:sculpting-in-concurrency}.

\textbf{Chapter 4} presents the relationship between certain non-interleaving models of concurrency. Specifically, we will relate higher-dimensional automata and ST-structures as well as ST-structures and Chu spaces. Higher-dimensional automata are not good at identifying events as show in Example \ref{exp:asymmetric-conflict}, but can be faithfully interpreted as ST-structures to identify events. In Example \ref{exp:asymmetric-conflict}, isomorphic higher-dimensional automata are interpreted as non-isomorphic ST-structures. Hence, there is not an embedding, that is, an injective morphism, from $\allST$ to $\allHDA$. We also show in Figure \ref{fig:Unfolding-HDA} that non-isomorphic higher-dimensional automata interpreted as ST-structures become isomorphic ST-structures, meaning there is not an embedding from $\allHDA$ to $\allST$. Hence, higher-dimensional automata are neither more, or less, expressive than ST-structures. We investigate the relationship between Chu spaces and ST-structure to better understand the state-event duality.

\textbf{Chapter 5} presents the method "\emph{sculpting}" which has not been studied for higher-dimensional automata before. We investigate the relationship between sculptures and ST-structures to tighten the correlation between ST-structures and higher-dimensional automata, which was left open in \cite{Johansen16STstruct}. We develop an algorithm to decide whether an higher-dimensional automaton can be sculpted or not, and show several simple examples of acyclic higher-dimensional automata which are not sculptures. We believe that this contradicts Pratt's conjecture.



% Introduces the notion of sculpting and defining it as Sculptures. Then comparing it to HDA and shows its ability to be equivalent with ST-structures and Chu spaces. There are however HDAs which can't be sculpted which we will show as part of one of the last sections of the thesis.

\textbf{Chapter 6} presents some concluding remarks, summarizes the contributions of the thesis and offers some ideas that can be pursued in future work.

%\ctlong{Include also the necessary examples, theorem/proofs and definitions from Topology, Category theory to build the theorem of Sculptures + proof}

%\ctlong{You should have many more theorems than definitions,, and many more examples than theorems.}

%\ctlong{We will add the mathematical background as an appendix, topology, category theory and }
