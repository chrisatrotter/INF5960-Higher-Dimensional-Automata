\section{The contributions of this thesis}
    Part of this thesis is dedicated to giving an introduction to concurrency models, where we focus on non-interleaving models of concurrency capable of capturing the independence of arbitrary number of events. In order to achieve this, we have included a review of several existing models of concurrency in Chapter \ref{chap:traditional-models-for-concurrency} and \ref{chap:An introduction to models for true concurrency}, including transition systems, event structures, configuration structures, asynchronous transition systems, higher dimensional automata, ST-structure and Chu spaces. This part does not represent new work in itself, and the relationships between the models in Chapter \ref{chap:Relationship with other models of true concurrency} have been investigated and are known in the geometric models of concurrency literature.
    
    On the other hand, we show the ideas the above mentioned models of concurrency have contributed to understanding both higher-dimensional automata and sculptures. In Chapter \ref{chap:traditional-models-for-concurrency}, we expend some effort to explain the difference between interleaving and non-interleaving models, and the need for non-interleaving models to incrementally develop concurrent systems by action refinement. Also, we investigate the event-state duality of non-interleaving models  to better understanding the current research being done. Specifically, investigating event-state duality in higher-dimensional automata by ST-structures and Chu spaces.
    
    The main contribution is the method of sculpting that is presented in Chapter \ref{chap:sculpting-in-concurrency}. Sculpting is a method of modelling concurrent behaviour of higher-dimensional automata. We investigate a conjecture posed by Vaughan Pratt in \cite{Pratt00Sculptures}. The conjecture is whether any higher-dimensional automata can be obtained using the sculpting method. We first define precisely the sculpting method, following again the intuition of Pratt that sculpting is similar to subalgebras. We develop an algorithm to decide whether an higher-dimensional automaton can be sculpted or not, and show several simple examples of \emph{acyclic} higher-dimensional automata which are not sculptures.
    
    We attempt to precisely identify the class of higher-dimensional automata which by sculpting can identify events. From the work by Johansen \cite{Johansen16STstruct}, we known higher-dimensional automata are in general not good at identifying events. However, as we show in this thesis, sculptures are well suited to represent the events, thus overcoming the problem identified in \cite[Figure 5]{Johansen16STstruct}. For example, the asymmetric conflict cannot be represented as a higher-dimensional automata, but can be faithfully represented as a sculpture or as an ST-structure. Going further, sculptures are one response to the event-state duality of Pratt \cite{Pratt02eventStateDuality} in higher-dimensional automata.
    
   