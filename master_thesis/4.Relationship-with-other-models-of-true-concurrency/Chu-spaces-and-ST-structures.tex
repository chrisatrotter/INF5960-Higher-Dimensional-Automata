\section{Chu spaces and ST-structures}
\label{sec:Chu-spaces-and-ST-structures}
    Chu spaces can be viewed in various equivalent ways, as mentioned in Section \ref{sec:Chu-spaces}. A Chu space can be viewed as a matrix with entries from $K$, where rows represent the events from $E$ and columns represent the configurations in $X$.
    
    ST-structures capture the "\emph{during}" aspect in the event-based setting, extending configuration structures with this notion. We consider Chu spaces $K= \mathbf{3}= \{0, \executing , 1\}$ with the order $0 < \executing\ < 1$, introducing the value $\executing$ to stand for \textit{during}, or \textit{in transition}. We see that Chu space over $\mathbf{3}$ is similar to the sculpting way of looking at Chu spaces as mentioned at the end of Section \ref{sec:Chu-spaces}. Hence, revealing the intrinsic symmetry of events and states. Chu space over $\mathbf{3}$ is also called a triadic chu space.


    \begin{definition}[Translations between ST and Chu]
        \label{def:ST-to-Chu}
        For an ST-structure $\ST$ over $E$ construct $\chu{E}{X}^{\ST}$ the associated Chu space over $\mathbf{3}$ with $E$ the set of events from $\ST$, and $X\subseteq \mathbf3^{E}$ containing for each ST-configuration $(\mathcal{S},\mathcal{T})$ the state $x^{(\mathcal{S},\mathcal{T})}\in X$ formed by assigning to each $e\in E$:
  
        \begin{itemize}
            \item $e \rightarrow 0$ if $e\not\in \mathcal{S}$ and $e\not\in \mathcal{T}$;
            \item $e \rightarrow \executing$ if $e\in \mathcal{S}$ and $e\not\in \mathcal{T}$;
            \item $e \rightarrow 1$ if $e\in \mathcal{S}$ and $e\in \mathcal{T}$.
        \end{itemize}
  
        The possibility that $e\notin \mathcal{S}$ and $e\in \mathcal{T}$ is dismissed by the requirement $\mathcal{T} \subseteq \mathcal{S}$ of ST-configurations.  Call this mapping $\stintochu(\mathcal{S},\mathcal{T})$ when applied to a ST-configuration and $\stintochu(\ST)$ when applied to an ST-structure.
  
        The other way, translate a Chu space $(E,X)$ into a ST-structure over $E$ with one ST-configuration $(\mathcal{S},\mathcal{T})^{x}$ for each state $x\in X$ using the inverse of the above mapping.  We use $\chuintost(x)$ for the ST-configuration obtained from the event listing $x$. For example, for an event listing $x=(1,\executing,\executing,0)$ make the ST-configuration $\chuintost(x)=(\{e_{1},e_{2},e_{3}\},\{e_{1}\})$ where the last event $e_{4}$ does not appear neither in the first nor the second set of the ST-configuration.
    \end{definition}

    \begin{proposition}[{\cite[Section 3.4]{Johansen16STstruct}}]
        \label{prop:ST-structure-Chu-over-3}
        ST-structures are isomorphic to Chu spaces over $\mathbf{3}$, such that $\chuintost(\stintochu(\ST)) \isomorphic \ST$.
        %  by virtue of the translations from Definition~\ref{def:ST-to-Chu}.
    \end{proposition}
    
    Thus, a ST-configuration can be seen as a listing/tuple with values from $\mathbf{3}$; which exact listing of the events $E$ is irrelevant once fixed.  Therefore, when we later use ST-configurations to \emph{label} cells of an \HDA, we can alternatively use the Chu spaces notation, interchangeably.

    \begin{lemma}
        For any ST-structure $\ST$ the Chu space $\stintochu(\ST)$ is \emph{extenssional}, meaning that no two states are identical, that is, $\forall x,x'.\exists e:x(e)\neq x'(e)$.
    \end{lemma}

    \begin{proof}
        In short, since ST-structures work with sets, we have that in the set of ST-configurations there are no two ST-configurations that are the same, then the states produced by $\stintochu$ would also be different by the virtue of the assignment from Definition~\ref{def:ST-to-Chu} which associates unique events valuation to a ST-configuration.

         In detail, for any $x^{(\mathcal{S},\mathcal{T})}\neq x^{(\mathcal{S}',\mathcal{T}')}$ they are created from some different $(\mathcal{S},\mathcal{T})\neq (\mathcal{S}',\mathcal{T}')$, which implies one of the two cases:
        
        \begin{enumerate}
            \item When $\mathcal{S} \neq \mathcal{S}'$ then pick some $e\in \mathcal{S}$ such  that $e\notin \mathcal{S}'$ (or the other way around if needed) then the states generated by $\stintochu$ would have the valuations: $x^{(\mathcal{S},\mathcal{T})}(e)\in\{\executing,1\}$ and $x^{(\mathcal{S}',\mathcal{T}')}(e)=0$, thus making them different.
            \item When $\mathcal{S} = \mathcal{S}'$ but $\mathcal{T} \neq \mathcal{T}'$ the pick some $e \in \mathcal{T}$ such that $e \notin \mathcal{T}'$ (or the other way around if needed) then the states would have  $x^{(\mathcal{S},\mathcal{T})}(e)=1$ (because $e \in \mathcal{T} \subseteq \mathcal{S}$) and $x^{(\mathcal{S}',\mathcal{T}')}(e) = \executing$, thus making them different.
        \end{enumerate}
    \end{proof}
    
    In Section \ref{sec:Chu-spaces}, a morphism between Chu spaces ($E, X$) and ($A, Y$) is defined to be a pair ($f, g$) of maps $f : E \to A$ and $g: Y \to X$ that satisfy the adjointness condition. We can extend $\stintochu$ to a functor between the two categories $\categoryST$ and $\categoryChu$\ by defining its application to morphisms as $\stintochu(f)=(f,g)$ with $g$ defined in terms of $f$ using the equation~\refeq{categoryChu1}.

    \begin{lemma}
        The $\stintochu$ is a functor between $\categoryST$ and $\categoryChu$.
    \end{lemma}

    \begin{proof}
        Since the Chu-spaces generated by $\stintochu$ are extensional, then we define $g$ by defining what the values of all the events are from $A$ in that state from $X$ as $g(y)(e)=y(f(e))$. This now makes the two mapings $(f,g)$ to respect the adjointness condition of the category $\categoryChu$, that is, to be a proper morphism in this category.
    \end{proof}

    Note that \chuintost\ cannot be a functor because the Chu-transforms are too weak
    % make too few relations between events of states of the two Chu-spaces
    to allow us to prove the configuration-preserving property of the ST-morphisms.