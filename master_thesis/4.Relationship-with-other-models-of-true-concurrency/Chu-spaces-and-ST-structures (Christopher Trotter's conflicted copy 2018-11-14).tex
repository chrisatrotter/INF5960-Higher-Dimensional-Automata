\section{Chu spaces and ST-structures}
\label{sec:Chu-spaces-and-ST-structures}
    Chu spaces can be viewed in various equivalent ways, as mentioned in Section \ref{sec:Chu-spaces}. A Chu space can be viewed as a matrix with entries from $K$, where rows represent the events from $E$ and columns represent the configurations in $X$.
    
    ST-structures capture the "\emph{during}" aspect in the event-based setting, extending configuration structures with this notion. We consider Chu spaces $K= \mathbf{3}= \{0, \executing , 1\}$ with the order $0 < \executing\ < 1$, introducing the value $\executing$ to stand for \textit{during}, or \textit{in transition}. We see that Chu space over $\mathbf{3}$ is similar to the sculpting way of looking at Chu spaces as mentioned at the end of Section \ref{sec:Chu-spaces}. Hence, revealing the intrinsic symmetry of events and states. Chu space over $\mathbf{3}$ is also called a triadic chu space.


    \begin{definition}[Translations between ST and Chu]
        \label{def:ST-to-Chu}
        For an ST-structure $\ST$ over $E$ construct $\chu{A}{X}^{\ST}$ the associated Chu space over $\mathbf{3}$ with $A=E$ the set of events from $\ST$, and $X\subseteq \mathbf3^{E}$ containing for each ST-configuration $(\mathcal{S},\mathcal{T})$ the state $x^{(\mathcal{S},\mathcal{T})}\in X$ formed by assigning to each $e\in E$:
  
        \begin{itemize}
            \item $e \rightarrow 0$ if $e\not\in \mathcal{S}$ and $e\not\in \mathcal{T}$;
            \item $e \rightarrow \executing$ if $e\in \mathcal{S}$ and $e\not\in \mathcal{T}$;
            \item $e \rightarrow 1$ if $e\in \mathcal{S}$ and $e\in \mathcal{T}$.
        \end{itemize}
  
        The possibility that $e\notin \mathcal{S}$ and $e\in \mathcal{T}$ is dismissed by the requirement $\mathcal{T} \subseteq \mathcal{S}$ of ST-configurations.  Call this mapping $\stintochu(\mathcal{S},\mathcal{T})$ when applied to an ST-configuration and $\stintochu(\ST)$ when applied to an ST-structure.
  
        The other way, translate a Chu space $(A,X)$ into an ST-structure over $A$ with one ST-configuration $(\mathcal{S},\mathcal{T})^{x}$ for each state $x\in X$ using the inverse of the above mapping.  We use $\chuintost(x)$ for the ST-configuration obtained from the event listing $x$. For example, for an event listing $x=(1,\executing,\executing,0)$ make the ST-configuration $\chuintost(x)=(\{e_{1},e_{2},e_{3}\},\{e_{1}\})$ where the last event $e_{4}$ does not appear neither in the first nor the second set of the ST-configuration.
    \end{definition}

    \begin{proposition}[{\cite[Section 3.4]{Johansen16STstruct}}]
        \label{prop:ST-structure-Chu-over-3}
        ST-structures are isomorphic to Chu spaces over $\mathbf{3}$, such that $\chuintost(\stintochu(\ST)) \isomorphic \ST$.
        %  by virtue of the translations from Definition~\ref{def:ST-to-Chu}.
    \end{proposition}
    
    Thus, an ST-configuration can be seen as a listing/tuple with values from $\mathbf{3}$; which exact listing of the events $E$ is irrelevant once fixed.  Therefore, when we later use ST-configurations to \emph{label} cells of an \HDA, we can alternatively use the Chu spaces notation, interchangeably.

    \begin{lemma}
        For any ST-structure $\ST$ the Chu space $\stintochu(\ST)$ is \emph{extenssional}, meaning that no two states are identical, that is, $\forall x,x'.\exists e:x(e)\neq x'(e)$.
    \end{lemma}

    \begin{proof}
        In short, since ST-structures work with sets, we have that in the set of ST-configurations there are no two ST-configurations that are the same, then the states produced by $\stintochu$ would also be different by the virtue of the assignment from Definition~\ref{def:ST-to-Chu} which associates unique events valuation to an ST-configuration.

         In detail, for any $x^{(\mathcal{S},\mathcal{T})}\neq x^{(\mathcal{S}',\mathcal{T}')}$ they are created from some different $(\mathcal{S},\mathcal{T})\neq (\mathcal{S}',\mathcal{T}')$, which implies one of the two cases:
        
        \begin{enumerate}
            \item When $\mathcal{S} \neq \mathcal{S}'$ then pick some $e\in \mathcal{S}$ such  that $e\notin \mathcal{S}'$ (or the other way around if needed) then the states generated by $\stintochu$ would have the valuations: $x^{(\mathcal{S},\mathcal{T})}(e)\in\{\executing,1\}$ and $x^{(\mathcal{S}',\mathcal{T}')}(e)=0$, thus making them different.
            \item When $\mathcal{S} = \mathcal{S}'$ but $\mathcal{T} \neq \mathcal{T}'$ the pick some $e \in \mathcal{T}$ such that $e \notin \mathcal{T}'$ (or the other way around if needed) then the states would have  $x^{(\mathcal{S},\mathcal{T})}(e)=1$ (because $e \in \mathcal{T} \subseteq \mathcal{S}$) and $x^{(\mathcal{S}',\mathcal{T}')}(e) = \executing$, thus making them different.
        \end{enumerate}
    \end{proof}

    %The category of Chu spaces, $\categoryChu$, has morphisms (called Chu transforms in \cite[Chapter 4]{gupta94phd_Chu}) between Chu spaces $(E,X)$ and $(A,Y)$ defined to be a pair $(f,g)$ of maps $f:E\rightarrow A$ and $g:Y\rightarrow X$ that satisfy the following two equations (called adjointness conditions in \cite[Chapter 4]{gupta94phd_Chu}) for all $e$ and for all $y$:

    %\begin{equation}
    %    g(y)(e) = y(f(e))\label{categoryChu1}
    %\end{equation} 

    %\begin{equation}
    %    e(g(y)) = f(e)(y)\label{categoryChu2}
    %\end{equation} 

    % 
    % We extend $\stintochu$ and \chuintost\ to functors between the two categories $\categoryST$ and $\categoryChu$\ by defining their application to morphisms as follows.
    % \begin{enumerate}
    % \item Take $\stintochu(f)=(f,g)$ with $g$ defined in terms of $f$ (because $\stintochu$ returns extenssional Chu spaces) using the equation~\refeq{categoryChu1}.
    % \item Take $\chuintost(f,g)=f$.
    % \end{enumerate}
    %
    
    In Section \ref{sec:Chu-spaces}, a morphism between Chu spaces ($E, X$) and ($A, Y$) is defined to be a pair ($f, g$) of maps $f : E \to A$ and $g: Y \to X$ that satisfy the adjointness condition. We can extend $\stintochu$ to a functor between the two categories $\categoryST$ and $\categoryChu$\ by defining its application to morphisms as $\stintochu(f)=(f,g)$ with $g$ defined in terms of $f$ using the equation~\refeq{categoryChu1}.

    \begin{lemma}
        The $\stintochu$ is a functor between $\categoryST$ and $\categoryChu$.
    \end{lemma}

    \begin{proof}
        Since the Chu-spaces generated by $\stintochu$ are extensional, then we define $g$ by defining what the values of all the events are from $A$ in that state from $X$ as $g(y)(e)=y(f(e))$. This now makes the two mapings $(f,g)$ to respect the adjointness condition of the category $\categoryChu$, that is, to be a proper morphism in this category.
    \end{proof}

    Note that \chuintost\ cannot be a functor because the Chu-transforms are too weak
    % make too few relations between events of states of the two Chu-spaces
    to allow us to prove the configuration-preserving property of the ST-morphisms.

    %\begin{lemma}
    %    $\chuintost$ is NOT a functor, that is, for any two $(E,X)$ and $(A,Y)$ with a morphism $(f,g)$ between them, then $\chuintost(f,g)$ \emph{cannot be proven to be a} morphism between $\chuintost(E,X)$ and $\chuintost(A,Y)$.
    %\end{lemma}

    %\begin{proof}
    %    Take $\chuintost(f,g)=f$.

    %    We need to prove that $f$ preserves configurations. Take some arbitrary $(\mathcal{S},\mathcal{T})^{x}\in\chuintost(E,X)$ and show that $f((\mathcal{S},\mathcal{T})^{x})\in\chuintost(A,Y)$. This means that we need to find some $y\in Y$ such that $f((\mathcal{S},\mathcal{T})^{x})=(\mathcal{S},\mathcal{T})^{y}$. Let us name $(\mathcal{S},\mathcal{T})^{x}=(\mathcal{S}^{x},\mathcal{T}^{x})$ and then have $f((\mathcal{S},\mathcal{T})^{x})=(f(\mathcal{S}^{x}),f(\mathcal{T}^{x}))$. Then our proof goal is rewritten as find $y$ such that $(f(\mathcal{S}^{x}),f(\mathcal{T}^{x})) = (\mathcal{S}^{y},\mathcal{T}^{y})$.

    %    If $g$ is surjective, then it means there exists $g^{-1}$ the right identity of $g$, that is, making $g(g^{-1}(x))=x$. Then we can define the above $y=g^{-1}(x)$ and the equation~\refeq{categoryChu1} gives us the equality $x(e) = y(f(e))$ for all $e$.

    %    The proof cannot be completed because not all events from $A$ can be related through $f$, and thus there are events outside the image of $f$ for which we do not know their valuations in the state $y$. This means we cannot know how a state $y$ is transformed into a $\mathcal{S}^{y}$ only by looking at the states $x$ and the maping of events $f$.

   % \end{proof}

    %\begin{lemma}
    %    When the Chu spaces are \emph{separable} \cite{Pratt02eventStateDuality} (called T0 in \cite{gupta94phd_Chu}), that is, that no two events are the same $\forall e,e'.\exists x:e(x)\neq e'(x)$ or $x(e)\neq x(e')$, then for a morphism $(f,g):(E,X)\rightarrow (A,Y)$ the fact that $g$ is surjective implies that $f$ is injective.
    %\end{lemma}

    %\begin{proof}
    %    Assume that $f$ is not injective, that is, $\exists e,e': f(e)=f(e')$. Then apply the equation~\refeq{categoryChu1} for all states $y$ to have $g(y)(e) = y(f(e)) = y(f(e')) = g(y)(e')$. Since $g$ is surjective it means that we reach all states $x$ and thus the above violates separability for $(E,X)$, since $\forall x: x(e)=x(e')$.
    %\end{proof}



    %\ulilong{Extend this to an isomorphism of categories? (What a Chu-morphisms?)}

    %\cjlong{We cannot do this, and there is no nice condition that we can put on the Chu category. All that we could do is to provide a functor form the $\categoryST$ to $\categoryChu$.}

    %An ST-configuration can be seen as a listing/tuple with values from $\mathbf{3}$, and what exact listing of the events $E$ is irrelevant once fixed.  Therefore, when we later use ST-configurations to \emph{label} cells of an $\HDA$, we can alternatively use the Chu spaces notation, interchangeably.

    %The results of \cite{Johansen16STstruct} work for ST-structures under the closure restriction: $\mbox{ if } (\mathcal{S},\mathcal{T})\in ST \mbox{ then }(\mathcal{S},\mathcal{S})\in ST$.  The same restriction can be imposed on Chu spaces as: $\mbox{ if } x\in X \mbox{ then } \exists y\in X \mbox{\ such\ that\ } x\!\!\downarrow_{0}=y\!\!\downarrow_{0} \mbox{ and } y\!\!\downarrow_{\executing}=\emptyset$. However, there is a more relaxed closure that captures the same intuitive purpose of ensuring that events that are started eventually are terminated: $ \mbox{ if } (\mathcal{S},\mathcal{T})\in ST \mbox{ then }\exists \mathcal{S}' \mbox{\ such\ that\ } \mathcal{S} \subseteq \mathcal{S}' \wedge (\mathcal{S}',\mathcal{S}')\in ST.  $
    %On Chu spaces this corresponds to the more natural restriction that: for any even that is ever in $\executing$, there also exists a state with it being terminate, that is, any row of the Chu matrix is either all $0$s or it contains both $\executing$ and $1$.

    %\cjlong{Not sure if all this paragraph discussion is needed. We maybe leave for another paper where this relaxation would be studied.}