In this chapter we show the relationship between non-interleaving models of concurrency reviewed in the previous chapter. Specifically, we will relate higher-dimensional automata and ST-structures as well as ST-structures and Chu spaces. The relationship between Chu spaces and higher-dimensional automata has been investigated by Pratt in \cite{Pratt02eventStateDuality}, and the relationship between ST-structures and higher-dimensional automata has been investigated by Johansen in \cite{Johansen16STstruct}. One of our motivations is to investigate an open problem of \cite{Johansen16STstruct} where there was no embedding between ST-structures and higher-dimensional automata. We attempt to investigate the relationships of the mentioned non-interleaving models of concurrency to better understand the event-state duality of Pratt \cite{Pratt02eventStateDuality}.

Higher-dimensional automata are not good at identifying events as show in Example \ref{exp:asymmetric-conflict}, but can be faithfully interpreted as ST-structures to identify events. In Example \ref{exp:asymmetric-conflict}, isomorphic higher-dimensional automata are interpreted as non-isomorphic ST-structures. Hence, there is not an embedding, that is, an injective morphism, from $\allST$ to $\allHDA$. We also show in Figure \ref{fig:Unfolding-HDA} that non-isomorphic higher-dimensional automata interpreted as ST-structures become isomorphic ST-structures, meaning there is not an embedding from $\allHDA$ to $\allST$. Hence, higher-dimensional automata are neither more, or less, expressive than ST-structures.

We will consider Chu spaces as a way to reconcile the relationship between higher-dimensional automata and ST-structures, which was left as an open problem in \cite{Johansen16STstruct}. Chu spaces are able to translate automata and schedules into each other by the Chu duality. Chu spaces are one response to the event-state duality, where automata and schedules are represented symmetrically. 

Higher-dimensional automata and Chu spaces have been investigated by Pratt \cite{Pratt02eventStateDuality}. The relationship between ST-structures and Chu spaces has been investigated by Johansen \cite{Johansen16STstruct}. However, we will present the relationship between ST-structures and Chu spaces to better answer the event-state duality question.
